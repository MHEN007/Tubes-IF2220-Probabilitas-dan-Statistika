\documentclass[11pt]{article}

    \usepackage[breakable]{tcolorbox}
    \usepackage{parskip} % Stop auto-indenting (to mimic markdown behaviour)
    

    % Basic figure setup, for now with no caption control since it's done
    % automatically by Pandoc (which extracts ![](path) syntax from Markdown).
    \usepackage{graphicx}
    % Maintain compatibility with old templates. Remove in nbconvert 6.0
    \let\Oldincludegraphics\includegraphics
    % Ensure that by default, figures have no caption (until we provide a
    % proper Figure object with a Caption API and a way to capture that
    % in the conversion process - todo).
    \usepackage{caption}
    \DeclareCaptionFormat{nocaption}{}
    \captionsetup{format=nocaption,aboveskip=0pt,belowskip=0pt}

    \usepackage{float}
    \floatplacement{figure}{H} % forces figures to be placed at the correct location
    \usepackage{xcolor} % Allow colors to be defined
    \usepackage{enumerate} % Needed for markdown enumerations to work
    \usepackage{geometry} % Used to adjust the document margins
    \usepackage{amsmath} % Equations
    \usepackage{amssymb} % Equations
    \usepackage{textcomp} % defines textquotesingle
    % Hack from http://tex.stackexchange.com/a/47451/13684:
    \AtBeginDocument{%
        \def\PYZsq{\textquotesingle}% Upright quotes in Pygmentized code
    }
    \usepackage{upquote} % Upright quotes for verbatim code
    \usepackage{eurosym} % defines \euro

    \usepackage{iftex}
    \ifPDFTeX
        \usepackage[T1]{fontenc}
        \IfFileExists{alphabeta.sty}{
              \usepackage{alphabeta}
          }{
              \usepackage[mathletters]{ucs}
              \usepackage[utf8x]{inputenc}
          }
    \else
        \usepackage{fontspec}
        \usepackage{unicode-math}
    \fi

    \usepackage{fancyvrb} % verbatim replacement that allows latex
    \usepackage{grffile} % extends the file name processing of package graphics
                         % to support a larger range
    \makeatletter % fix for old versions of grffile with XeLaTeX
    \@ifpackagelater{grffile}{2019/11/01}
    {
      % Do nothing on new versions
    }
    {
      \def\Gread@@xetex#1{%
        \IfFileExists{"\Gin@base".bb}%
        {\Gread@eps{\Gin@base.bb}}%
        {\Gread@@xetex@aux#1}%
      }
    }
    \makeatother
    \usepackage[Export]{adjustbox} % Used to constrain images to a maximum size
    \adjustboxset{max size={0.9\linewidth}{0.9\paperheight}}

    % The hyperref package gives us a pdf with properly built
    % internal navigation ('pdf bookmarks' for the table of contents,
    % internal cross-reference links, web links for URLs, etc.)
    \usepackage{hyperref}
    % The default LaTeX title has an obnoxious amount of whitespace. By default,
    % titling removes some of it. It also provides customization options.
    \usepackage{titling}
    \usepackage{longtable} % longtable support required by pandoc >1.10
    \usepackage{booktabs}  % table support for pandoc > 1.12.2
    \usepackage{array}     % table support for pandoc >= 2.11.3
    \usepackage{calc}      % table minipage width calculation for pandoc >= 2.11.1
    \usepackage[inline]{enumitem} % IRkernel/repr support (it uses the enumerate* environment)
    \usepackage[normalem]{ulem} % ulem is needed to support strikethroughs (\sout)
                                % normalem makes italics be italics, not underlines
    \usepackage{mathrsfs}
    

    
    % Colors for the hyperref package
    \definecolor{urlcolor}{rgb}{0,.145,.698}
    \definecolor{linkcolor}{rgb}{.71,0.21,0.01}
    \definecolor{citecolor}{rgb}{.12,.54,.11}

    % ANSI colors
    \definecolor{ansi-black}{HTML}{3E424D}
    \definecolor{ansi-black-intense}{HTML}{282C36}
    \definecolor{ansi-red}{HTML}{E75C58}
    \definecolor{ansi-red-intense}{HTML}{B22B31}
    \definecolor{ansi-green}{HTML}{00A250}
    \definecolor{ansi-green-intense}{HTML}{007427}
    \definecolor{ansi-yellow}{HTML}{DDB62B}
    \definecolor{ansi-yellow-intense}{HTML}{B27D12}
    \definecolor{ansi-blue}{HTML}{208FFB}
    \definecolor{ansi-blue-intense}{HTML}{0065CA}
    \definecolor{ansi-magenta}{HTML}{D160C4}
    \definecolor{ansi-magenta-intense}{HTML}{A03196}
    \definecolor{ansi-cyan}{HTML}{60C6C8}
    \definecolor{ansi-cyan-intense}{HTML}{258F8F}
    \definecolor{ansi-white}{HTML}{C5C1B4}
    \definecolor{ansi-white-intense}{HTML}{A1A6B2}
    \definecolor{ansi-default-inverse-fg}{HTML}{FFFFFF}
    \definecolor{ansi-default-inverse-bg}{HTML}{000000}

    % common color for the border for error outputs.
    \definecolor{outerrorbackground}{HTML}{FFDFDF}

    % commands and environments needed by pandoc snippets
    % extracted from the output of `pandoc -s`
    \providecommand{\tightlist}{%
      \setlength{\itemsep}{0pt}\setlength{\parskip}{0pt}}
    \DefineVerbatimEnvironment{Highlighting}{Verbatim}{commandchars=\\\{\}}
    % Add ',fontsize=\small' for more characters per line
    \newenvironment{Shaded}{}{}
    \newcommand{\KeywordTok}[1]{\textcolor[rgb]{0.00,0.44,0.13}{\textbf{{#1}}}}
    \newcommand{\DataTypeTok}[1]{\textcolor[rgb]{0.56,0.13,0.00}{{#1}}}
    \newcommand{\DecValTok}[1]{\textcolor[rgb]{0.25,0.63,0.44}{{#1}}}
    \newcommand{\BaseNTok}[1]{\textcolor[rgb]{0.25,0.63,0.44}{{#1}}}
    \newcommand{\FloatTok}[1]{\textcolor[rgb]{0.25,0.63,0.44}{{#1}}}
    \newcommand{\CharTok}[1]{\textcolor[rgb]{0.25,0.44,0.63}{{#1}}}
    \newcommand{\StringTok}[1]{\textcolor[rgb]{0.25,0.44,0.63}{{#1}}}
    \newcommand{\CommentTok}[1]{\textcolor[rgb]{0.38,0.63,0.69}{\textit{{#1}}}}
    \newcommand{\OtherTok}[1]{\textcolor[rgb]{0.00,0.44,0.13}{{#1}}}
    \newcommand{\AlertTok}[1]{\textcolor[rgb]{1.00,0.00,0.00}{\textbf{{#1}}}}
    \newcommand{\FunctionTok}[1]{\textcolor[rgb]{0.02,0.16,0.49}{{#1}}}
    \newcommand{\RegionMarkerTok}[1]{{#1}}
    \newcommand{\ErrorTok}[1]{\textcolor[rgb]{1.00,0.00,0.00}{\textbf{{#1}}}}
    \newcommand{\NormalTok}[1]{{#1}}

    % Additional commands for more recent versions of Pandoc
    \newcommand{\ConstantTok}[1]{\textcolor[rgb]{0.53,0.00,0.00}{{#1}}}
    \newcommand{\SpecialCharTok}[1]{\textcolor[rgb]{0.25,0.44,0.63}{{#1}}}
    \newcommand{\VerbatimStringTok}[1]{\textcolor[rgb]{0.25,0.44,0.63}{{#1}}}
    \newcommand{\SpecialStringTok}[1]{\textcolor[rgb]{0.73,0.40,0.53}{{#1}}}
    \newcommand{\ImportTok}[1]{{#1}}
    \newcommand{\DocumentationTok}[1]{\textcolor[rgb]{0.73,0.13,0.13}{\textit{{#1}}}}
    \newcommand{\AnnotationTok}[1]{\textcolor[rgb]{0.38,0.63,0.69}{\textbf{\textit{{#1}}}}}
    \newcommand{\CommentVarTok}[1]{\textcolor[rgb]{0.38,0.63,0.69}{\textbf{\textit{{#1}}}}}
    \newcommand{\VariableTok}[1]{\textcolor[rgb]{0.10,0.09,0.49}{{#1}}}
    \newcommand{\ControlFlowTok}[1]{\textcolor[rgb]{0.00,0.44,0.13}{\textbf{{#1}}}}
    \newcommand{\OperatorTok}[1]{\textcolor[rgb]{0.40,0.40,0.40}{{#1}}}
    \newcommand{\BuiltInTok}[1]{{#1}}
    \newcommand{\ExtensionTok}[1]{{#1}}
    \newcommand{\PreprocessorTok}[1]{\textcolor[rgb]{0.74,0.48,0.00}{{#1}}}
    \newcommand{\AttributeTok}[1]{\textcolor[rgb]{0.49,0.56,0.16}{{#1}}}
    \newcommand{\InformationTok}[1]{\textcolor[rgb]{0.38,0.63,0.69}{\textbf{\textit{{#1}}}}}
    \newcommand{\WarningTok}[1]{\textcolor[rgb]{0.38,0.63,0.69}{\textbf{\textit{{#1}}}}}


    % Define a nice break command that doesn't care if a line doesn't already
    % exist.
    \def\br{\hspace*{\fill} \\* }
    % Math Jax compatibility definitions
    \def\gt{>}
    \def\lt{<}
    \let\Oldtex\TeX
    \let\Oldlatex\LaTeX
    \renewcommand{\TeX}{\textrm{\Oldtex}}
    \renewcommand{\LaTeX}{\textrm{\Oldlatex}}
    % Document parameters
    % Document title
    \title{K03-T1-IF2220-13521004-13521007}
    
    
    
    
    
% Pygments definitions
\makeatletter
\def\PY@reset{\let\PY@it=\relax \let\PY@bf=\relax%
    \let\PY@ul=\relax \let\PY@tc=\relax%
    \let\PY@bc=\relax \let\PY@ff=\relax}
\def\PY@tok#1{\csname PY@tok@#1\endcsname}
\def\PY@toks#1+{\ifx\relax#1\empty\else%
    \PY@tok{#1}\expandafter\PY@toks\fi}
\def\PY@do#1{\PY@bc{\PY@tc{\PY@ul{%
    \PY@it{\PY@bf{\PY@ff{#1}}}}}}}
\def\PY#1#2{\PY@reset\PY@toks#1+\relax+\PY@do{#2}}

\@namedef{PY@tok@w}{\def\PY@tc##1{\textcolor[rgb]{0.73,0.73,0.73}{##1}}}
\@namedef{PY@tok@c}{\let\PY@it=\textit\def\PY@tc##1{\textcolor[rgb]{0.25,0.50,0.50}{##1}}}
\@namedef{PY@tok@cp}{\def\PY@tc##1{\textcolor[rgb]{0.74,0.48,0.00}{##1}}}
\@namedef{PY@tok@k}{\let\PY@bf=\textbf\def\PY@tc##1{\textcolor[rgb]{0.00,0.50,0.00}{##1}}}
\@namedef{PY@tok@kp}{\def\PY@tc##1{\textcolor[rgb]{0.00,0.50,0.00}{##1}}}
\@namedef{PY@tok@kt}{\def\PY@tc##1{\textcolor[rgb]{0.69,0.00,0.25}{##1}}}
\@namedef{PY@tok@o}{\def\PY@tc##1{\textcolor[rgb]{0.40,0.40,0.40}{##1}}}
\@namedef{PY@tok@ow}{\let\PY@bf=\textbf\def\PY@tc##1{\textcolor[rgb]{0.67,0.13,1.00}{##1}}}
\@namedef{PY@tok@nb}{\def\PY@tc##1{\textcolor[rgb]{0.00,0.50,0.00}{##1}}}
\@namedef{PY@tok@nf}{\def\PY@tc##1{\textcolor[rgb]{0.00,0.00,1.00}{##1}}}
\@namedef{PY@tok@nc}{\let\PY@bf=\textbf\def\PY@tc##1{\textcolor[rgb]{0.00,0.00,1.00}{##1}}}
\@namedef{PY@tok@nn}{\let\PY@bf=\textbf\def\PY@tc##1{\textcolor[rgb]{0.00,0.00,1.00}{##1}}}
\@namedef{PY@tok@ne}{\let\PY@bf=\textbf\def\PY@tc##1{\textcolor[rgb]{0.82,0.25,0.23}{##1}}}
\@namedef{PY@tok@nv}{\def\PY@tc##1{\textcolor[rgb]{0.10,0.09,0.49}{##1}}}
\@namedef{PY@tok@no}{\def\PY@tc##1{\textcolor[rgb]{0.53,0.00,0.00}{##1}}}
\@namedef{PY@tok@nl}{\def\PY@tc##1{\textcolor[rgb]{0.63,0.63,0.00}{##1}}}
\@namedef{PY@tok@ni}{\let\PY@bf=\textbf\def\PY@tc##1{\textcolor[rgb]{0.60,0.60,0.60}{##1}}}
\@namedef{PY@tok@na}{\def\PY@tc##1{\textcolor[rgb]{0.49,0.56,0.16}{##1}}}
\@namedef{PY@tok@nt}{\let\PY@bf=\textbf\def\PY@tc##1{\textcolor[rgb]{0.00,0.50,0.00}{##1}}}
\@namedef{PY@tok@nd}{\def\PY@tc##1{\textcolor[rgb]{0.67,0.13,1.00}{##1}}}
\@namedef{PY@tok@s}{\def\PY@tc##1{\textcolor[rgb]{0.73,0.13,0.13}{##1}}}
\@namedef{PY@tok@sd}{\let\PY@it=\textit\def\PY@tc##1{\textcolor[rgb]{0.73,0.13,0.13}{##1}}}
\@namedef{PY@tok@si}{\let\PY@bf=\textbf\def\PY@tc##1{\textcolor[rgb]{0.73,0.40,0.53}{##1}}}
\@namedef{PY@tok@se}{\let\PY@bf=\textbf\def\PY@tc##1{\textcolor[rgb]{0.73,0.40,0.13}{##1}}}
\@namedef{PY@tok@sr}{\def\PY@tc##1{\textcolor[rgb]{0.73,0.40,0.53}{##1}}}
\@namedef{PY@tok@ss}{\def\PY@tc##1{\textcolor[rgb]{0.10,0.09,0.49}{##1}}}
\@namedef{PY@tok@sx}{\def\PY@tc##1{\textcolor[rgb]{0.00,0.50,0.00}{##1}}}
\@namedef{PY@tok@m}{\def\PY@tc##1{\textcolor[rgb]{0.40,0.40,0.40}{##1}}}
\@namedef{PY@tok@gh}{\let\PY@bf=\textbf\def\PY@tc##1{\textcolor[rgb]{0.00,0.00,0.50}{##1}}}
\@namedef{PY@tok@gu}{\let\PY@bf=\textbf\def\PY@tc##1{\textcolor[rgb]{0.50,0.00,0.50}{##1}}}
\@namedef{PY@tok@gd}{\def\PY@tc##1{\textcolor[rgb]{0.63,0.00,0.00}{##1}}}
\@namedef{PY@tok@gi}{\def\PY@tc##1{\textcolor[rgb]{0.00,0.63,0.00}{##1}}}
\@namedef{PY@tok@gr}{\def\PY@tc##1{\textcolor[rgb]{1.00,0.00,0.00}{##1}}}
\@namedef{PY@tok@ge}{\let\PY@it=\textit}
\@namedef{PY@tok@gs}{\let\PY@bf=\textbf}
\@namedef{PY@tok@gp}{\let\PY@bf=\textbf\def\PY@tc##1{\textcolor[rgb]{0.00,0.00,0.50}{##1}}}
\@namedef{PY@tok@go}{\def\PY@tc##1{\textcolor[rgb]{0.53,0.53,0.53}{##1}}}
\@namedef{PY@tok@gt}{\def\PY@tc##1{\textcolor[rgb]{0.00,0.27,0.87}{##1}}}
\@namedef{PY@tok@err}{\def\PY@bc##1{{\setlength{\fboxsep}{\string -\fboxrule}\fcolorbox[rgb]{1.00,0.00,0.00}{1,1,1}{\strut ##1}}}}
\@namedef{PY@tok@kc}{\let\PY@bf=\textbf\def\PY@tc##1{\textcolor[rgb]{0.00,0.50,0.00}{##1}}}
\@namedef{PY@tok@kd}{\let\PY@bf=\textbf\def\PY@tc##1{\textcolor[rgb]{0.00,0.50,0.00}{##1}}}
\@namedef{PY@tok@kn}{\let\PY@bf=\textbf\def\PY@tc##1{\textcolor[rgb]{0.00,0.50,0.00}{##1}}}
\@namedef{PY@tok@kr}{\let\PY@bf=\textbf\def\PY@tc##1{\textcolor[rgb]{0.00,0.50,0.00}{##1}}}
\@namedef{PY@tok@bp}{\def\PY@tc##1{\textcolor[rgb]{0.00,0.50,0.00}{##1}}}
\@namedef{PY@tok@fm}{\def\PY@tc##1{\textcolor[rgb]{0.00,0.00,1.00}{##1}}}
\@namedef{PY@tok@vc}{\def\PY@tc##1{\textcolor[rgb]{0.10,0.09,0.49}{##1}}}
\@namedef{PY@tok@vg}{\def\PY@tc##1{\textcolor[rgb]{0.10,0.09,0.49}{##1}}}
\@namedef{PY@tok@vi}{\def\PY@tc##1{\textcolor[rgb]{0.10,0.09,0.49}{##1}}}
\@namedef{PY@tok@vm}{\def\PY@tc##1{\textcolor[rgb]{0.10,0.09,0.49}{##1}}}
\@namedef{PY@tok@sa}{\def\PY@tc##1{\textcolor[rgb]{0.73,0.13,0.13}{##1}}}
\@namedef{PY@tok@sb}{\def\PY@tc##1{\textcolor[rgb]{0.73,0.13,0.13}{##1}}}
\@namedef{PY@tok@sc}{\def\PY@tc##1{\textcolor[rgb]{0.73,0.13,0.13}{##1}}}
\@namedef{PY@tok@dl}{\def\PY@tc##1{\textcolor[rgb]{0.73,0.13,0.13}{##1}}}
\@namedef{PY@tok@s2}{\def\PY@tc##1{\textcolor[rgb]{0.73,0.13,0.13}{##1}}}
\@namedef{PY@tok@sh}{\def\PY@tc##1{\textcolor[rgb]{0.73,0.13,0.13}{##1}}}
\@namedef{PY@tok@s1}{\def\PY@tc##1{\textcolor[rgb]{0.73,0.13,0.13}{##1}}}
\@namedef{PY@tok@mb}{\def\PY@tc##1{\textcolor[rgb]{0.40,0.40,0.40}{##1}}}
\@namedef{PY@tok@mf}{\def\PY@tc##1{\textcolor[rgb]{0.40,0.40,0.40}{##1}}}
\@namedef{PY@tok@mh}{\def\PY@tc##1{\textcolor[rgb]{0.40,0.40,0.40}{##1}}}
\@namedef{PY@tok@mi}{\def\PY@tc##1{\textcolor[rgb]{0.40,0.40,0.40}{##1}}}
\@namedef{PY@tok@il}{\def\PY@tc##1{\textcolor[rgb]{0.40,0.40,0.40}{##1}}}
\@namedef{PY@tok@mo}{\def\PY@tc##1{\textcolor[rgb]{0.40,0.40,0.40}{##1}}}
\@namedef{PY@tok@ch}{\let\PY@it=\textit\def\PY@tc##1{\textcolor[rgb]{0.25,0.50,0.50}{##1}}}
\@namedef{PY@tok@cm}{\let\PY@it=\textit\def\PY@tc##1{\textcolor[rgb]{0.25,0.50,0.50}{##1}}}
\@namedef{PY@tok@cpf}{\let\PY@it=\textit\def\PY@tc##1{\textcolor[rgb]{0.25,0.50,0.50}{##1}}}
\@namedef{PY@tok@c1}{\let\PY@it=\textit\def\PY@tc##1{\textcolor[rgb]{0.25,0.50,0.50}{##1}}}
\@namedef{PY@tok@cs}{\let\PY@it=\textit\def\PY@tc##1{\textcolor[rgb]{0.25,0.50,0.50}{##1}}}

\def\PYZbs{\char`\\}
\def\PYZus{\char`\_}
\def\PYZob{\char`\{}
\def\PYZcb{\char`\}}
\def\PYZca{\char`\^}
\def\PYZam{\char`\&}
\def\PYZlt{\char`\<}
\def\PYZgt{\char`\>}
\def\PYZsh{\char`\#}
\def\PYZpc{\char`\%}
\def\PYZdl{\char`\$}
\def\PYZhy{\char`\-}
\def\PYZsq{\char`\'}
\def\PYZdq{\char`\"}
\def\PYZti{\char`\~}
% for compatibility with earlier versions
\def\PYZat{@}
\def\PYZlb{[}
\def\PYZrb{]}
\makeatother


    % For linebreaks inside Verbatim environment from package fancyvrb.
    \makeatletter
        \newbox\Wrappedcontinuationbox
        \newbox\Wrappedvisiblespacebox
        \newcommand*\Wrappedvisiblespace {\textcolor{red}{\textvisiblespace}}
        \newcommand*\Wrappedcontinuationsymbol {\textcolor{red}{\llap{\tiny$\m@th\hookrightarrow$}}}
        \newcommand*\Wrappedcontinuationindent {3ex }
        \newcommand*\Wrappedafterbreak {\kern\Wrappedcontinuationindent\copy\Wrappedcontinuationbox}
        % Take advantage of the already applied Pygments mark-up to insert
        % potential linebreaks for TeX processing.
        %        {, <, #, %, $, ' and ": go to next line.
        %        _, }, ^, &, >, - and ~: stay at end of broken line.
        % Use of \textquotesingle for straight quote.
        \newcommand*\Wrappedbreaksatspecials {%
            \def\PYGZus{\discretionary{\char`\_}{\Wrappedafterbreak}{\char`\_}}%
            \def\PYGZob{\discretionary{}{\Wrappedafterbreak\char`\{}{\char`\{}}%
            \def\PYGZcb{\discretionary{\char`\}}{\Wrappedafterbreak}{\char`\}}}%
            \def\PYGZca{\discretionary{\char`\^}{\Wrappedafterbreak}{\char`\^}}%
            \def\PYGZam{\discretionary{\char`\&}{\Wrappedafterbreak}{\char`\&}}%
            \def\PYGZlt{\discretionary{}{\Wrappedafterbreak\char`\<}{\char`\<}}%
            \def\PYGZgt{\discretionary{\char`\>}{\Wrappedafterbreak}{\char`\>}}%
            \def\PYGZsh{\discretionary{}{\Wrappedafterbreak\char`\#}{\char`\#}}%
            \def\PYGZpc{\discretionary{}{\Wrappedafterbreak\char`\%}{\char`\%}}%
            \def\PYGZdl{\discretionary{}{\Wrappedafterbreak\char`\$}{\char`\$}}%
            \def\PYGZhy{\discretionary{\char`\-}{\Wrappedafterbreak}{\char`\-}}%
            \def\PYGZsq{\discretionary{}{\Wrappedafterbreak\textquotesingle}{\textquotesingle}}%
            \def\PYGZdq{\discretionary{}{\Wrappedafterbreak\char`\"}{\char`\"}}%
            \def\PYGZti{\discretionary{\char`\~}{\Wrappedafterbreak}{\char`\~}}%
        }
        % Some characters . , ; ? ! / are not pygmentized.
        % This macro makes them "active" and they will insert potential linebreaks
        \newcommand*\Wrappedbreaksatpunct {%
            \lccode`\~`\.\lowercase{\def~}{\discretionary{\hbox{\char`\.}}{\Wrappedafterbreak}{\hbox{\char`\.}}}%
            \lccode`\~`\,\lowercase{\def~}{\discretionary{\hbox{\char`\,}}{\Wrappedafterbreak}{\hbox{\char`\,}}}%
            \lccode`\~`\;\lowercase{\def~}{\discretionary{\hbox{\char`\;}}{\Wrappedafterbreak}{\hbox{\char`\;}}}%
            \lccode`\~`\:\lowercase{\def~}{\discretionary{\hbox{\char`\:}}{\Wrappedafterbreak}{\hbox{\char`\:}}}%
            \lccode`\~`\?\lowercase{\def~}{\discretionary{\hbox{\char`\?}}{\Wrappedafterbreak}{\hbox{\char`\?}}}%
            \lccode`\~`\!\lowercase{\def~}{\discretionary{\hbox{\char`\!}}{\Wrappedafterbreak}{\hbox{\char`\!}}}%
            \lccode`\~`\/\lowercase{\def~}{\discretionary{\hbox{\char`\/}}{\Wrappedafterbreak}{\hbox{\char`\/}}}%
            \catcode`\.\active
            \catcode`\,\active
            \catcode`\;\active
            \catcode`\:\active
            \catcode`\?\active
            \catcode`\!\active
            \catcode`\/\active
            \lccode`\~`\~
        }
    \makeatother

    \let\OriginalVerbatim=\Verbatim
    \makeatletter
    \renewcommand{\Verbatim}[1][1]{%
        %\parskip\z@skip
        \sbox\Wrappedcontinuationbox {\Wrappedcontinuationsymbol}%
        \sbox\Wrappedvisiblespacebox {\FV@SetupFont\Wrappedvisiblespace}%
        \def\FancyVerbFormatLine ##1{\hsize\linewidth
            \vtop{\raggedright\hyphenpenalty\z@\exhyphenpenalty\z@
                \doublehyphendemerits\z@\finalhyphendemerits\z@
                \strut ##1\strut}%
        }%
        % If the linebreak is at a space, the latter will be displayed as visible
        % space at end of first line, and a continuation symbol starts next line.
        % Stretch/shrink are however usually zero for typewriter font.
        \def\FV@Space {%
            \nobreak\hskip\z@ plus\fontdimen3\font minus\fontdimen4\font
            \discretionary{\copy\Wrappedvisiblespacebox}{\Wrappedafterbreak}
            {\kern\fontdimen2\font}%
        }%

        % Allow breaks at special characters using \PYG... macros.
        \Wrappedbreaksatspecials
        % Breaks at punctuation characters . , ; ? ! and / need catcode=\active
        \OriginalVerbatim[#1,codes*=\Wrappedbreaksatpunct]%
    }
    \makeatother

    % Exact colors from NB
    \definecolor{incolor}{HTML}{303F9F}
    \definecolor{outcolor}{HTML}{D84315}
    \definecolor{cellborder}{HTML}{CFCFCF}
    \definecolor{cellbackground}{HTML}{F7F7F7}

    % prompt
    \makeatletter
    \newcommand{\boxspacing}{\kern\kvtcb@left@rule\kern\kvtcb@boxsep}
    \makeatother
    \newcommand{\prompt}[4]{
        {\ttfamily\llap{{\color{#2}[#3]:\hspace{3pt}#4}}\vspace{-\baselineskip}}
    }
    

    
    % Prevent overflowing lines due to hard-to-break entities
    \sloppy
    % Setup hyperref package
    \hypersetup{
      breaklinks=true,  % so long urls are correctly broken across lines
      colorlinks=true,
      urlcolor=urlcolor,
      linkcolor=linkcolor,
      citecolor=citecolor,
      }
    % Slightly bigger margins than the latex defaults
    
    \geometry{verbose,tmargin=1in,bmargin=1in,lmargin=1in,rmargin=1in}
    
    

\begin{document}
    
    \maketitle
    
    

    
    \hypertarget{tugas-besar-1---if2220-probabilitas-dan-statistika}{%
\section{Tugas Besar 1 - IF2220 Probabilitas dan
Statistika}\label{tugas-besar-1---if2220-probabilitas-dan-statistika}}

\hypertarget{dibuat-oleh}{%
\subsection{Dibuat Oleh}\label{dibuat-oleh}}

\begin{longtable}[]{@{}ll@{}}
\toprule\noalign{}
NIM & Nama \\
\midrule\noalign{}
\endhead
\bottomrule\noalign{}
\endlastfoot
13521004 & Henry Anand Septian Radityo \\
13521007 & Matthew Mahendra \\
\end{longtable}

    \hypertarget{setup}{%
\section{Setup}\label{setup}}

    \begin{tcolorbox}[breakable, size=fbox, boxrule=1pt, pad at break*=1mm,colback=cellbackground, colframe=cellborder]
\prompt{In}{incolor}{1}{\boxspacing}
\begin{Verbatim}[commandchars=\\\{\}]
\PY{k+kn}{import} \PY{n+nn}{pandas} \PY{k}{as} \PY{n+nn}{pd}
\PY{k+kn}{import} \PY{n+nn}{numpy} \PY{k}{as} \PY{n+nn}{np}
\PY{k+kn}{import} \PY{n+nn}{matplotlib}\PY{n+nn}{.}\PY{n+nn}{pyplot} \PY{k}{as} \PY{n+nn}{plt}
\PY{k+kn}{import} \PY{n+nn}{seaborn} \PY{k}{as} \PY{n+nn}{sns}
\PY{k+kn}{import} \PY{n+nn}{scipy}\PY{n+nn}{.}\PY{n+nn}{stats}
\PY{k+kn}{from} \PY{n+nn}{scipy} \PY{k+kn}{import} \PY{n}{stats}
\PY{k+kn}{from} \PY{n+nn}{scipy}\PY{n+nn}{.}\PY{n+nn}{stats} \PY{k+kn}{import} \PY{n}{norm}
\PY{k+kn}{from} \PY{n+nn}{scipy}\PY{n+nn}{.}\PY{n+nn}{stats} \PY{k+kn}{import} \PY{n}{shapiro}\PY{p}{,} \PY{n}{probplot}
\PY{k+kn}{from} \PY{n+nn}{IPython}\PY{n+nn}{.}\PY{n+nn}{display} \PY{k+kn}{import} \PY{n}{display}\PY{p}{,} \PY{n}{Markdown}\PY{p}{,} \PY{n}{Latex}

\PY{n}{df} \PY{o}{=} \PY{n}{pd}\PY{o}{.}\PY{n}{read\PYZus{}csv}\PY{p}{(}\PY{l+s+s1}{\PYZsq{}}\PY{l+s+s1}{anggur.csv}\PY{l+s+s1}{\PYZsq{}}\PY{p}{)}

\PY{c+c1}{\PYZsh{} Cleanup untuk data NaN jika ada}
\PY{n}{df} \PY{o}{=} \PY{n}{df}\PY{o}{.}\PY{n}{dropna}\PY{p}{(}\PY{p}{)}
\end{Verbatim}
\end{tcolorbox}

    \hypertarget{soal}{%
\section{SOAL}\label{soal}}

    Diberikan sebuah data anggur.csv yang dapat diakses pada utas berikut:
\href{https://github.com/MHEN2606/Tubes-IF2220-Probabilitas-dan-Statistika/blob/main/anggur.csv}{Dataset
Tugas Besar IF2220} merupakan data metrik kualitas wine (minuman anggur)
yang mengandung 12 kolom sebagai berikut: 1. fixed acidity 2. volatile
acidity 3. citric acid 4. residual sugar 5. chlorides 6. free sulfur
dioxide 7. total sulfur dioxide 8. density 9. pH 10. sulphates 11.
alcohol 12. quality

Kolom 1-11 adalah kolom atribut (non-target), sedangkan kolom 12 adalah
kolom target. Anda diminta untuk melakukan analisis statistika sebagai
berikut: 1. Menulis deskripsi statistika (Descriptive Statistics) dari
semua kolom pada data yang bersifat numerik, terdiri dari mean, median,
modus, standar deviasi, variansi, range, nilai minimum, maksimum,
kuartil, IQR, skewness dan kurtosis. Boleh juga ditambahkan deskripsi
lain. 2. Membuat Visualisasi plot distribusi, dalam bentuk histogram dan
boxplot untuk setiap kolom numerik. Berikan uraian penjelasan kondisi
setiap kolom berdasarkan kedua plot tersebut. 3. Menentukan setiap kolom
numerik berdistribusi normal atau tidak. Gunakan normality test yang
dikaitkan dengan histogram plot. 4. Melakukan test hipotesis 1 sampel, -
Nilai rata-rata pH di atas 3.29? - Nilai rata-rata Residual Sugar tidak
sama dengan 2.50? - Nilai rata-rata 150 baris pertama kolom sulphates
bukan 0.65? - Nilai rata-rata total sulfur dioxide di bawah 35 -
Proporsi nilai total Sulfat Dioxide yang lebih dari 40, adalah tidak
sama dengan 50\% ?

\begin{enumerate}
\def\labelenumi{\arabic{enumi}.}
\setcounter{enumi}{4}
\tightlist
\item
  Melakukan test hipotesis 2 sampel,
\end{enumerate}

\begin{itemize}
\tightlist
\item
  Data kolom fixed acidity dibagi 2 sama rata: bagian awal dan bagian
  akhir kolom. Benarkah rata-rata kedua bagian tersebut sama?
\item
  Data kolom chlorides dibagi 2 sama rata: bagian awal dan bagian akhir
  kolom. Benarkah rata-rata bagian awal lebih besar daripada bagian
  akhir sebesar 0.001?
\item
  Benarkah rata-rata sampel 25 baris pertama kolom Volatile Acidity sama
  dengan rata-rata 25 baris pertama kolom Sulphates ?
\item
  Bagian awal kolom residual sugar memiliki variansi yang sama dengan
  bagian akhirnya?
\item
  Proporsi nilai setengah bagian awal alcohol yang lebih dari 7, adalah
  lebih besar daripada, proporsi nilai yang sama di setengah bagian
  akhir alcohol?
\end{itemize}

    \begin{center}\rule{0.5\linewidth}{0.5pt}\end{center}

    \hypertarget{soal-1}{%
\section{SOAL 1}\label{soal-1}}

    Mean of Data

    \begin{tcolorbox}[breakable, size=fbox, boxrule=1pt, pad at break*=1mm,colback=cellbackground, colframe=cellborder]
\prompt{In}{incolor}{2}{\boxspacing}
\begin{Verbatim}[commandchars=\\\{\}]
\PY{c+c1}{\PYZsh{} Mean dari setiap kolom}
\PY{n}{df}\PY{o}{.}\PY{n}{mean}\PY{p}{(}\PY{p}{)}
\end{Verbatim}
\end{tcolorbox}

            \begin{tcolorbox}[breakable, size=fbox, boxrule=.5pt, pad at break*=1mm, opacityfill=0]
\prompt{Out}{outcolor}{2}{\boxspacing}
\begin{Verbatim}[commandchars=\\\{\}]
fixed acidity            7.152530
volatile acidity         0.520839
citric acid              0.270517
residual sugar           2.567104
chlorides                0.081195
free sulfur dioxide     14.907679
total sulfur dioxide    40.290150
density                  0.995925
pH                       3.303610
sulphates                0.598390
alcohol                 10.592280
quality                  7.958000
dtype: float64
\end{Verbatim}
\end{tcolorbox}
        
    Median of Data

    \begin{tcolorbox}[breakable, size=fbox, boxrule=1pt, pad at break*=1mm,colback=cellbackground, colframe=cellborder]
\prompt{In}{incolor}{3}{\boxspacing}
\begin{Verbatim}[commandchars=\\\{\}]
\PY{c+c1}{\PYZsh{} Median dari setiap kolom}
\PY{n}{df}\PY{o}{.}\PY{n}{median}\PY{p}{(}\PY{p}{)}
\end{Verbatim}
\end{tcolorbox}

            \begin{tcolorbox}[breakable, size=fbox, boxrule=.5pt, pad at break*=1mm, opacityfill=0]
\prompt{Out}{outcolor}{3}{\boxspacing}
\begin{Verbatim}[commandchars=\\\{\}]
fixed acidity            7.150000
volatile acidity         0.524850
citric acid              0.272200
residual sugar           2.519430
chlorides                0.082167
free sulfur dioxide     14.860346
total sulfur dioxide    40.190000
density                  0.996000
pH                       3.300000
sulphates                0.595000
alcohol                 10.610000
quality                  8.000000
dtype: float64
\end{Verbatim}
\end{tcolorbox}
        
    Standar Deviation of Data

    \begin{tcolorbox}[breakable, size=fbox, boxrule=1pt, pad at break*=1mm,colback=cellbackground, colframe=cellborder]
\prompt{In}{incolor}{4}{\boxspacing}
\begin{Verbatim}[commandchars=\\\{\}]
\PY{c+c1}{\PYZsh{} Standar Deviation dari setiap kolom}
\PY{n}{df}\PY{o}{.}\PY{n}{std}\PY{p}{(}\PY{p}{)}
\end{Verbatim}
\end{tcolorbox}

            \begin{tcolorbox}[breakable, size=fbox, boxrule=.5pt, pad at break*=1mm, opacityfill=0]
\prompt{Out}{outcolor}{4}{\boxspacing}
\begin{Verbatim}[commandchars=\\\{\}]
fixed acidity           1.201598
volatile acidity        0.095848
citric acid             0.049098
residual sugar          0.987915
chlorides               0.020111
free sulfur dioxide     4.888100
total sulfur dioxide    9.965767
density                 0.002020
pH                      0.104875
sulphates               0.100819
alcohol                 1.510706
quality                 0.902802
dtype: float64
\end{Verbatim}
\end{tcolorbox}
        
    Variance of Data

    \begin{tcolorbox}[breakable, size=fbox, boxrule=1pt, pad at break*=1mm,colback=cellbackground, colframe=cellborder]
\prompt{In}{incolor}{5}{\boxspacing}
\begin{Verbatim}[commandchars=\\\{\}]
\PY{c+c1}{\PYZsh{} Variance}
\PY{n}{df}\PY{o}{.}\PY{n}{std}\PY{p}{(}\PY{p}{)}\PY{o}{*}\PY{o}{*}\PY{l+m+mi}{2}
\end{Verbatim}
\end{tcolorbox}

            \begin{tcolorbox}[breakable, size=fbox, boxrule=.5pt, pad at break*=1mm, opacityfill=0]
\prompt{Out}{outcolor}{5}{\boxspacing}
\begin{Verbatim}[commandchars=\\\{\}]
fixed acidity            1.443837
volatile acidity         0.009187
citric acid              0.002411
residual sugar           0.975977
chlorides                0.000404
free sulfur dioxide     23.893519
total sulfur dioxide    99.316519
density                  0.000004
pH                       0.010999
sulphates                0.010164
alcohol                  2.282233
quality                  0.815051
dtype: float64
\end{Verbatim}
\end{tcolorbox}
        
    Range of Data

    \begin{tcolorbox}[breakable, size=fbox, boxrule=1pt, pad at break*=1mm,colback=cellbackground, colframe=cellborder]
\prompt{In}{incolor}{6}{\boxspacing}
\begin{Verbatim}[commandchars=\\\{\}]
\PY{c+c1}{\PYZsh{} Range}
\PY{n}{df}\PY{o}{.}\PY{n}{max}\PY{p}{(}\PY{p}{)} \PY{o}{\PYZhy{}} \PY{n}{df}\PY{o}{.}\PY{n}{min}\PY{p}{(}\PY{p}{)}
\end{Verbatim}
\end{tcolorbox}

            \begin{tcolorbox}[breakable, size=fbox, boxrule=.5pt, pad at break*=1mm, opacityfill=0]
\prompt{Out}{outcolor}{6}{\boxspacing}
\begin{Verbatim}[commandchars=\\\{\}]
fixed acidity            8.170000
volatile acidity         0.665200
citric acid              0.292900
residual sugar           5.518200
chlorides                0.125635
free sulfur dioxide     27.267847
total sulfur dioxide    66.810000
density                  0.013800
pH                       0.740000
sulphates                0.670000
alcohol                  8.990000
quality                  5.000000
dtype: float64
\end{Verbatim}
\end{tcolorbox}
        
    Quantiles of Data

    \begin{tcolorbox}[breakable, size=fbox, boxrule=1pt, pad at break*=1mm,colback=cellbackground, colframe=cellborder]
\prompt{In}{incolor}{7}{\boxspacing}
\begin{Verbatim}[commandchars=\\\{\}]
\PY{c+c1}{\PYZsh{} Q1}
\PY{n}{df}\PY{o}{.}\PY{n}{quantile}\PY{p}{(}\PY{l+m+mf}{0.25}\PY{p}{)}
\end{Verbatim}
\end{tcolorbox}

            \begin{tcolorbox}[breakable, size=fbox, boxrule=.5pt, pad at break*=1mm, opacityfill=0]
\prompt{Out}{outcolor}{7}{\boxspacing}
\begin{Verbatim}[commandchars=\\\{\}]
fixed acidity            6.377500
volatile acidity         0.456100
citric acid              0.237800
residual sugar           1.896330
chlorides                0.066574
free sulfur dioxide     11.426717
total sulfur dioxide    33.785000
density                  0.994600
pH                       3.230000
sulphates                0.530000
alcohol                  9.560000
quality                  7.000000
Name: 0.25, dtype: float64
\end{Verbatim}
\end{tcolorbox}
        
    \begin{tcolorbox}[breakable, size=fbox, boxrule=1pt, pad at break*=1mm,colback=cellbackground, colframe=cellborder]
\prompt{In}{incolor}{8}{\boxspacing}
\begin{Verbatim}[commandchars=\\\{\}]
\PY{c+c1}{\PYZsh{} Q2}
\PY{n}{df}\PY{o}{.}\PY{n}{quantile}\PY{p}{(}\PY{l+m+mf}{0.5}\PY{p}{)}
\end{Verbatim}
\end{tcolorbox}

            \begin{tcolorbox}[breakable, size=fbox, boxrule=.5pt, pad at break*=1mm, opacityfill=0]
\prompt{Out}{outcolor}{8}{\boxspacing}
\begin{Verbatim}[commandchars=\\\{\}]
fixed acidity            7.150000
volatile acidity         0.524850
citric acid              0.272200
residual sugar           2.519430
chlorides                0.082167
free sulfur dioxide     14.860346
total sulfur dioxide    40.190000
density                  0.996000
pH                       3.300000
sulphates                0.595000
alcohol                 10.610000
quality                  8.000000
Name: 0.5, dtype: float64
\end{Verbatim}
\end{tcolorbox}
        
    \begin{tcolorbox}[breakable, size=fbox, boxrule=1pt, pad at break*=1mm,colback=cellbackground, colframe=cellborder]
\prompt{In}{incolor}{9}{\boxspacing}
\begin{Verbatim}[commandchars=\\\{\}]
\PY{c+c1}{\PYZsh{} Q3}
\PY{n}{df}\PY{o}{.}\PY{n}{quantile}\PY{p}{(}\PY{l+m+mf}{0.75}\PY{p}{)}
\end{Verbatim}
\end{tcolorbox}

            \begin{tcolorbox}[breakable, size=fbox, boxrule=.5pt, pad at break*=1mm, opacityfill=0]
\prompt{Out}{outcolor}{9}{\boxspacing}
\begin{Verbatim}[commandchars=\\\{\}]
fixed acidity            8.000000
volatile acidity         0.585375
citric acid              0.302325
residual sugar           3.220873
chlorides                0.095312
free sulfur dioxide     18.313098
total sulfur dioxide    47.022500
density                  0.997200
pH                       3.370000
sulphates                0.670000
alcohol                 11.622500
quality                  9.000000
Name: 0.75, dtype: float64
\end{Verbatim}
\end{tcolorbox}
        
    Inter-Quantile Range of Data

    \begin{tcolorbox}[breakable, size=fbox, boxrule=1pt, pad at break*=1mm,colback=cellbackground, colframe=cellborder]
\prompt{In}{incolor}{10}{\boxspacing}
\begin{Verbatim}[commandchars=\\\{\}]
\PY{c+c1}{\PYZsh{} IQR}
\PY{n}{df}\PY{o}{.}\PY{n}{quantile}\PY{p}{(}\PY{l+m+mf}{0.75}\PY{p}{)} \PY{o}{\PYZhy{}} \PY{n}{df}\PY{o}{.}\PY{n}{quantile}\PY{p}{(}\PY{l+m+mf}{0.25}\PY{p}{)}
\end{Verbatim}
\end{tcolorbox}

            \begin{tcolorbox}[breakable, size=fbox, boxrule=.5pt, pad at break*=1mm, opacityfill=0]
\prompt{Out}{outcolor}{10}{\boxspacing}
\begin{Verbatim}[commandchars=\\\{\}]
fixed acidity            1.622500
volatile acidity         0.129275
citric acid              0.064525
residual sugar           1.324544
chlorides                0.028738
free sulfur dioxide      6.886381
total sulfur dioxide    13.237500
density                  0.002600
pH                       0.140000
sulphates                0.140000
alcohol                  2.062500
quality                  2.000000
dtype: float64
\end{Verbatim}
\end{tcolorbox}
        
    Skewness of Data

    \begin{tcolorbox}[breakable, size=fbox, boxrule=1pt, pad at break*=1mm,colback=cellbackground, colframe=cellborder]
\prompt{In}{incolor}{11}{\boxspacing}
\begin{Verbatim}[commandchars=\\\{\}]
\PY{c+c1}{\PYZsh{} Skewness}
\PY{n}{df}\PY{o}{.}\PY{n}{skew}\PY{p}{(}\PY{p}{)}
\end{Verbatim}
\end{tcolorbox}

            \begin{tcolorbox}[breakable, size=fbox, boxrule=.5pt, pad at break*=1mm, opacityfill=0]
\prompt{Out}{outcolor}{11}{\boxspacing}
\begin{Verbatim}[commandchars=\\\{\}]
fixed acidity          -0.028879
volatile acidity       -0.197699
citric acid            -0.045576
residual sugar          0.132638
chlorides              -0.051319
free sulfur dioxide     0.007130
total sulfur dioxide   -0.024060
density                -0.076883
pH                      0.147673
sulphates               0.149199
alcohol                -0.018991
quality                -0.089054
dtype: float64
\end{Verbatim}
\end{tcolorbox}
        
    Kurtosis of Data

    \begin{tcolorbox}[breakable, size=fbox, boxrule=1pt, pad at break*=1mm,colback=cellbackground, colframe=cellborder]
\prompt{In}{incolor}{12}{\boxspacing}
\begin{Verbatim}[commandchars=\\\{\}]
\PY{c+c1}{\PYZsh{} Kurtosis}
\PY{n}{df}\PY{o}{.}\PY{n}{kurtosis}\PY{p}{(}\PY{p}{)}
\end{Verbatim}
\end{tcolorbox}

            \begin{tcolorbox}[breakable, size=fbox, boxrule=.5pt, pad at break*=1mm, opacityfill=0]
\prompt{Out}{outcolor}{12}{\boxspacing}
\begin{Verbatim}[commandchars=\\\{\}]
fixed acidity          -0.019292
volatile acidity        0.161853
citric acid            -0.104679
residual sugar         -0.042980
chlorides              -0.246508
free sulfur dioxide    -0.364964
total sulfur dioxide    0.063950
density                 0.016366
pH                      0.080910
sulphates               0.064819
alcohol                -0.131732
quality                 0.108291
dtype: float64
\end{Verbatim}
\end{tcolorbox}
        
    Mode of Data

    \begin{tcolorbox}[breakable, size=fbox, boxrule=1pt, pad at break*=1mm,colback=cellbackground, colframe=cellborder]
\prompt{In}{incolor}{13}{\boxspacing}
\begin{Verbatim}[commandchars=\\\{\}]
\PY{n}{df}\PY{o}{.}\PY{n}{mode}\PY{p}{(}\PY{n}{numeric\PYZus{}only}\PY{o}{=}\PY{k+kc}{True}\PY{p}{,} \PY{n}{dropna}\PY{o}{=}\PY{k+kc}{True}\PY{p}{)}
\end{Verbatim}
\end{tcolorbox}

            \begin{tcolorbox}[breakable, size=fbox, boxrule=.5pt, pad at break*=1mm, opacityfill=0]
\prompt{Out}{outcolor}{13}{\boxspacing}
\begin{Verbatim}[commandchars=\\\{\}]
     fixed acidity  volatile acidity  citric acid  residual sugar  chlorides  \textbackslash{}
0             6.54            0.5546       0.3019        0.032555   0.015122
1              NaN               NaN          NaN        0.033333   0.020794
2              NaN               NaN          NaN        0.051774   0.024259
3              NaN               NaN          NaN        0.077156   0.027209
4              NaN               NaN          NaN        0.084744   0.032111
..             {\ldots}               {\ldots}          {\ldots}             {\ldots}        {\ldots}
995            NaN               NaN          NaN        5.210260   0.131425
996            NaN               NaN          NaN        5.217429   0.133656
997            NaN               NaN          NaN        5.252864   0.135368
998            NaN               NaN          NaN        5.299524   0.135790
999            NaN               NaN          NaN        5.550755   0.140758

     free sulfur dioxide  total sulfur dioxide  density    pH  sulphates  \textbackslash{}
0               0.194679                 35.20   0.9959  3.34       0.59
1               0.621628                 37.25   0.9961   NaN        NaN
2               0.860177                 39.64   0.9965   NaN        NaN
3               3.032139                 40.61   0.9970   NaN        NaN
4               3.129885                 41.05      NaN   NaN        NaN
..                   {\ldots}                   {\ldots}      {\ldots}   {\ldots}        {\ldots}
995            26.630490                   NaN      NaN   NaN        NaN
996            26.665773                   NaN      NaN   NaN        NaN
997            26.822626                   NaN      NaN   NaN        NaN
998            27.006307                   NaN      NaN   NaN        NaN
999            27.462525                   NaN      NaN   NaN        NaN

     alcohol  quality
0       9.86      8.0
1      10.31      NaN
2        NaN      NaN
3        NaN      NaN
4        NaN      NaN
..       {\ldots}      {\ldots}
995      NaN      NaN
996      NaN      NaN
997      NaN      NaN
998      NaN      NaN
999      NaN      NaN

[1000 rows x 12 columns]
\end{Verbatim}
\end{tcolorbox}
        
    \begin{tcolorbox}[breakable, size=fbox, boxrule=1pt, pad at break*=1mm,colback=cellbackground, colframe=cellborder]
\prompt{In}{incolor}{14}{\boxspacing}
\begin{Verbatim}[commandchars=\\\{\}]
\PY{c+c1}{\PYZsh{} Mode}
\PY{n}{df}\PY{p}{[}\PY{l+s+s1}{\PYZsq{}}\PY{l+s+s1}{fixed acidity}\PY{l+s+s1}{\PYZsq{}}\PY{p}{]}\PY{o}{.}\PY{n}{mode}\PY{p}{(}\PY{p}{)}

\PY{k}{for} \PY{n}{cols} \PY{o+ow}{in} \PY{n}{df}\PY{p}{:}
    \PY{n}{temp} \PY{o}{=} \PY{l+s+s2}{\PYZdq{}}\PY{l+s+s2}{Modus dari }\PY{l+s+s2}{\PYZdq{}} \PY{o}{+} \PY{n}{cols} \PY{o}{+} \PY{l+s+s2}{\PYZdq{}}\PY{l+s+s2}{:}\PY{l+s+s2}{\PYZdq{}}
    \PY{n}{display}\PY{p}{(}\PY{n}{Markdown}\PY{p}{(}\PY{l+s+sa}{f}\PY{l+s+s2}{\PYZdq{}}\PY{l+s+s2}{**}\PY{l+s+si}{\PYZob{}}\PY{n}{temp}\PY{l+s+si}{\PYZcb{}}\PY{l+s+s2}{**}\PY{l+s+s2}{\PYZdq{}}\PY{p}{)}\PY{p}{)}
    \PY{n}{modus} \PY{o}{=} \PY{p}{[}\PY{p}{]}
    \PY{k}{for} \PY{n}{i} \PY{o+ow}{in} \PY{n}{df}\PY{p}{[}\PY{n}{cols}\PY{p}{]}\PY{o}{.}\PY{n}{mode}\PY{p}{(}\PY{p}{)}\PY{p}{:}
        \PY{n}{modus}\PY{o}{.}\PY{n}{append}\PY{p}{(}\PY{n}{i}\PY{p}{)}
    \PY{n}{display}\PY{p}{(}\PY{n}{pd}\PY{o}{.}\PY{n}{DataFrame}\PY{p}{(}\PY{n}{modus}\PY{p}{)}\PY{p}{)}
\end{Verbatim}
\end{tcolorbox}

    \textbf{Modus dari fixed acidity:}

    
    
    \begin{Verbatim}[commandchars=\\\{\}]
      0
0  6.54
    \end{Verbatim}

    
    \textbf{Modus dari volatile acidity:}

    
    
    \begin{Verbatim}[commandchars=\\\{\}]
        0
0  0.5546
    \end{Verbatim}

    
    \textbf{Modus dari citric acid:}

    
    
    \begin{Verbatim}[commandchars=\\\{\}]
        0
0  0.3019
    \end{Verbatim}

    
    \textbf{Modus dari residual sugar:}

    
    
    \begin{Verbatim}[commandchars=\\\{\}]
            0
0    0.032555
1    0.033333
2    0.051774
3    0.077156
4    0.084744
..        {\ldots}
995  5.210260
996  5.217429
997  5.252864
998  5.299524
999  5.550755

[1000 rows x 1 columns]
    \end{Verbatim}

    
    \textbf{Modus dari chlorides:}

    
    
    \begin{Verbatim}[commandchars=\\\{\}]
            0
0    0.015122
1    0.020794
2    0.024259
3    0.027209
4    0.032111
..        {\ldots}
995  0.131425
996  0.133656
997  0.135368
998  0.135790
999  0.140758

[1000 rows x 1 columns]
    \end{Verbatim}

    
    \textbf{Modus dari free sulfur dioxide:}

    
    
    \begin{Verbatim}[commandchars=\\\{\}]
             0
0     0.194679
1     0.621628
2     0.860177
3     3.032139
4     3.129885
..         {\ldots}
995  26.630490
996  26.665773
997  26.822626
998  27.006307
999  27.462525

[1000 rows x 1 columns]
    \end{Verbatim}

    
    \textbf{Modus dari total sulfur dioxide:}

    
    
    \begin{Verbatim}[commandchars=\\\{\}]
       0
0  35.20
1  37.25
2  39.64
3  40.61
4  41.05
5  41.59
6  44.51
    \end{Verbatim}

    
    \textbf{Modus dari density:}

    
    
    \begin{Verbatim}[commandchars=\\\{\}]
        0
0  0.9959
1  0.9961
2  0.9965
3  0.9970
    \end{Verbatim}

    
    \textbf{Modus dari pH:}

    
    
    \begin{Verbatim}[commandchars=\\\{\}]
      0
0  3.34
    \end{Verbatim}

    
    \textbf{Modus dari sulphates:}

    
    
    \begin{Verbatim}[commandchars=\\\{\}]
      0
0  0.59
    \end{Verbatim}

    
    \textbf{Modus dari alcohol:}

    
    
    \begin{Verbatim}[commandchars=\\\{\}]
       0
0   9.86
1  10.31
    \end{Verbatim}

    
    \textbf{Modus dari quality:}

    
    
    \begin{Verbatim}[commandchars=\\\{\}]
   0
0  8
    \end{Verbatim}

    
    \begin{center}\rule{0.5\linewidth}{0.5pt}\end{center}

    \hypertarget{soal-2}{%
\section{SOAL 2}\label{soal-2}}

    Membuat Visualisasi plot distribusi, dalam bentuk histogram dan boxplot
untuk setiap kolom numerik. Berikan uraian penjelasan kondisi setiap
kolom berdasarkan kedua plot tersebut.

    \hypertarget{kolom-fixed-acidity}{%
\subsection{Kolom ``fixed acidity''}\label{kolom-fixed-acidity}}

    \begin{tcolorbox}[breakable, size=fbox, boxrule=1pt, pad at break*=1mm,colback=cellbackground, colframe=cellborder]
\prompt{In}{incolor}{15}{\boxspacing}
\begin{Verbatim}[commandchars=\\\{\}]
\PY{n}{var} \PY{o}{=} \PY{n}{df}\PY{p}{[}\PY{l+s+s1}{\PYZsq{}}\PY{l+s+s1}{fixed acidity}\PY{l+s+s1}{\PYZsq{}}\PY{p}{]}
\PY{n}{plt}\PY{o}{.}\PY{n}{figure}\PY{p}{(}\PY{n}{figsize}\PY{o}{=}\PY{p}{(}\PY{l+m+mi}{7}\PY{p}{,}\PY{l+m+mi}{6}\PY{p}{)}\PY{p}{)}
\PY{n}{fig}\PY{p}{,}\PY{n}{ax} \PY{o}{=} \PY{n}{plt}\PY{o}{.}\PY{n}{subplots}\PY{p}{(}\PY{l+m+mi}{1}\PY{p}{,}\PY{l+m+mi}{2}\PY{p}{)}
\PY{c+c1}{\PYZsh{} df[var].hist(ax = ax[0]) matplotlib}
\PY{n}{sns}\PY{o}{.}\PY{n}{histplot}\PY{p}{(}\PY{n}{var}\PY{p}{,} \PY{n}{ax}\PY{o}{=}\PY{n}{ax}\PY{p}{[}\PY{l+m+mi}{0}\PY{p}{]}\PY{p}{)}
\PY{n}{ax}\PY{p}{[}\PY{l+m+mi}{0}\PY{p}{]}\PY{o}{.}\PY{n}{set\PYZus{}title}\PY{p}{(}\PY{l+s+s2}{\PYZdq{}}\PY{l+s+s2}{Histogram fixed acidity}\PY{l+s+s2}{\PYZdq{}}\PY{p}{)}
\PY{n}{ax}\PY{p}{[}\PY{l+m+mi}{1}\PY{p}{]}\PY{o}{.}\PY{n}{set\PYZus{}title}\PY{p}{(}\PY{l+s+s2}{\PYZdq{}}\PY{l+s+s2}{Boxplot fixed acidity}\PY{l+s+s2}{\PYZdq{}}\PY{p}{)}
\PY{c+c1}{\PYZsh{} df.boxplot(var,ax=ax[1]) matplotlib}
\PY{n}{sns}\PY{o}{.}\PY{n}{boxplot}\PY{p}{(}\PY{n}{var}\PY{p}{,}\PY{n}{ax}\PY{o}{=}\PY{n}{ax}\PY{p}{[}\PY{l+m+mi}{1}\PY{p}{]}\PY{p}{)}
\PY{n}{plt}\PY{o}{.}\PY{n}{show}\PY{p}{(}\PY{p}{)}
\end{Verbatim}
\end{tcolorbox}

    
    \begin{Verbatim}[commandchars=\\\{\}]
<Figure size 700x600 with 0 Axes>
    \end{Verbatim}

    
    \begin{center}
    \adjustimage{max size={0.9\linewidth}{0.9\paperheight}}{output_34_1.png}
    \end{center}
    { \hspace*{\fill} \\}
    
    Dari histogram tersebut dapat dilihat bahwa persebaran data untuk kolom
fixed acidity cenderung terdistribusi secara merata apabila dilihat dari
grafik yang berbentuk simetris seperti lonceng dan titik tertinggi
berada di bagian tengah dari data dan rata-rata data yang ada di angka
7.15.

Grafik Box Plot juga menunjukkan data yang cenderung
\textbf{terdistribusi normal} yang ditunjukkan dari interquartile range
yang berada di tengah nilai minimum dan maksimum. Dari box plot juga
dapat dilihat terdapat beberapa data outlier yaitu data yang berada
diluar batas atas dan batas bawah. Terdapat 3 data yang berada di bawah
batas bawah dan 3 data berada diatas batas atas.

    \hypertarget{kolom-folatile-acidity}{%
\subsection{Kolom ``folatile acidity''}\label{kolom-folatile-acidity}}

    \begin{tcolorbox}[breakable, size=fbox, boxrule=1pt, pad at break*=1mm,colback=cellbackground, colframe=cellborder]
\prompt{In}{incolor}{16}{\boxspacing}
\begin{Verbatim}[commandchars=\\\{\}]
\PY{n}{var} \PY{o}{=} \PY{n}{df}\PY{p}{[}\PY{l+s+s1}{\PYZsq{}}\PY{l+s+s1}{volatile acidity}\PY{l+s+s1}{\PYZsq{}}\PY{p}{]}
\PY{n}{plt}\PY{o}{.}\PY{n}{figure}\PY{p}{(}\PY{n}{figsize}\PY{o}{=}\PY{p}{(}\PY{l+m+mi}{7}\PY{p}{,}\PY{l+m+mi}{6}\PY{p}{)}\PY{p}{)}
\PY{n}{fig}\PY{p}{,}\PY{n}{ax} \PY{o}{=} \PY{n}{plt}\PY{o}{.}\PY{n}{subplots}\PY{p}{(}\PY{l+m+mi}{1}\PY{p}{,}\PY{l+m+mi}{2}\PY{p}{)}
\PY{c+c1}{\PYZsh{} df[var].hist(ax = ax[0]) matplotlib}
\PY{n}{sns}\PY{o}{.}\PY{n}{histplot}\PY{p}{(}\PY{n}{var}\PY{p}{,} \PY{n}{ax}\PY{o}{=}\PY{n}{ax}\PY{p}{[}\PY{l+m+mi}{0}\PY{p}{]}\PY{p}{)}
\PY{n}{ax}\PY{p}{[}\PY{l+m+mi}{0}\PY{p}{]}\PY{o}{.}\PY{n}{set\PYZus{}title}\PY{p}{(}\PY{l+s+s2}{\PYZdq{}}\PY{l+s+s2}{Histogram volatile acidity}\PY{l+s+s2}{\PYZdq{}}\PY{p}{)}
\PY{n}{ax}\PY{p}{[}\PY{l+m+mi}{1}\PY{p}{]}\PY{o}{.}\PY{n}{set\PYZus{}title}\PY{p}{(}\PY{l+s+s2}{\PYZdq{}}\PY{l+s+s2}{Boxplot volatile acidity}\PY{l+s+s2}{\PYZdq{}}\PY{p}{)}
\PY{c+c1}{\PYZsh{} df.boxplot(var,ax=ax[1]) matplotlib}
\PY{n}{sns}\PY{o}{.}\PY{n}{boxplot}\PY{p}{(}\PY{n}{var}\PY{p}{,}\PY{n}{ax}\PY{o}{=}\PY{n}{ax}\PY{p}{[}\PY{l+m+mi}{1}\PY{p}{]}\PY{p}{)}
\PY{n}{plt}\PY{o}{.}\PY{n}{show}\PY{p}{(}\PY{p}{)}
\end{Verbatim}
\end{tcolorbox}

    
    \begin{Verbatim}[commandchars=\\\{\}]
<Figure size 700x600 with 0 Axes>
    \end{Verbatim}

    
    \begin{center}
    \adjustimage{max size={0.9\linewidth}{0.9\paperheight}}{output_37_1.png}
    \end{center}
    { \hspace*{\fill} \\}
    
    Dari histogram tersebut dapat dilihat bahwa persebaran data untuk kolom
folatile acidity memiliki persebaran paling banyak pada nilai di sekitar
0.5. Grafik Histogram terlihat simetris dengan persebaran yang cukup
merata.

Grafik Box Plot juga menunjukkan data outlier yaitu data yang berada
diluar batas atas dan batas bawah. Terdapat 6 data yang berada di bawah
batas bawah dan 1 data berada diatas batas atas.

    \hypertarget{kolom-citric-acid}{%
\subsection{Kolom ``citric acid''}\label{kolom-citric-acid}}

    \begin{tcolorbox}[breakable, size=fbox, boxrule=1pt, pad at break*=1mm,colback=cellbackground, colframe=cellborder]
\prompt{In}{incolor}{17}{\boxspacing}
\begin{Verbatim}[commandchars=\\\{\}]
\PY{n}{var} \PY{o}{=} \PY{n}{df}\PY{p}{[}\PY{l+s+s1}{\PYZsq{}}\PY{l+s+s1}{citric acid}\PY{l+s+s1}{\PYZsq{}}\PY{p}{]}
\PY{n}{plt}\PY{o}{.}\PY{n}{figure}\PY{p}{(}\PY{n}{figsize}\PY{o}{=}\PY{p}{(}\PY{l+m+mi}{7}\PY{p}{,}\PY{l+m+mi}{6}\PY{p}{)}\PY{p}{)}
\PY{n}{fig}\PY{p}{,}\PY{n}{ax} \PY{o}{=} \PY{n}{plt}\PY{o}{.}\PY{n}{subplots}\PY{p}{(}\PY{l+m+mi}{1}\PY{p}{,}\PY{l+m+mi}{2}\PY{p}{)}
\PY{c+c1}{\PYZsh{} df[var].hist(ax = ax[0]) matplotlib}
\PY{n}{sns}\PY{o}{.}\PY{n}{histplot}\PY{p}{(}\PY{n}{var}\PY{p}{,} \PY{n}{ax}\PY{o}{=}\PY{n}{ax}\PY{p}{[}\PY{l+m+mi}{0}\PY{p}{]}\PY{p}{)}
\PY{n}{ax}\PY{p}{[}\PY{l+m+mi}{0}\PY{p}{]}\PY{o}{.}\PY{n}{set\PYZus{}title}\PY{p}{(}\PY{l+s+s2}{\PYZdq{}}\PY{l+s+s2}{Histogram citric acid}\PY{l+s+s2}{\PYZdq{}}\PY{p}{)}
\PY{n}{ax}\PY{p}{[}\PY{l+m+mi}{1}\PY{p}{]}\PY{o}{.}\PY{n}{set\PYZus{}title}\PY{p}{(}\PY{l+s+s2}{\PYZdq{}}\PY{l+s+s2}{Boxplot citric acid}\PY{l+s+s2}{\PYZdq{}}\PY{p}{)}
\PY{c+c1}{\PYZsh{} df.boxplot(var,ax=ax[1]) matplotlib}
\PY{n}{sns}\PY{o}{.}\PY{n}{boxplot}\PY{p}{(}\PY{n}{var}\PY{p}{,}\PY{n}{ax}\PY{o}{=}\PY{n}{ax}\PY{p}{[}\PY{l+m+mi}{1}\PY{p}{]}\PY{p}{)}
\PY{n}{plt}\PY{o}{.}\PY{n}{show}\PY{p}{(}\PY{p}{)}
\end{Verbatim}
\end{tcolorbox}

    
    \begin{Verbatim}[commandchars=\\\{\}]
<Figure size 700x600 with 0 Axes>
    \end{Verbatim}

    
    \begin{center}
    \adjustimage{max size={0.9\linewidth}{0.9\paperheight}}{output_40_1.png}
    \end{center}
    { \hspace*{\fill} \\}
    
    Dari histogram tersebut dapat dilihat bahwa persebaran data untuk kolom
citric acid \textbf{terdistribusi normal}, apabila dilihat dari grafik
yang berbentuk simetris seperti lonceng dan titik tertinggi berada di
bagian tengah dari data dan rata-rata data yang ada di angka 0.27.
Distribusi normal juga dapat dilihat dari kurva bagian samping yang
memiliki frekuensi lebih sedikit dibanding dengan bagian tengah.

Grafik Box Plot juga menunjukkan data yang \textbf{terdistribusi normal}
yang ditunjukkan dari interquartile range yang berada di tengah nilai
minimum dan maksimum. Dari box plot juga dapat dilihat terdapat beberapa
data outlier yaitu data yang berada diluar batas atas dan batas bawah.
Terdapat 4 data yang berada di bawah batas bawah dan 2 data berada
diatas batas atas.

    \hypertarget{kolom-residual-sugar}{%
\subsection{Kolom ``residual sugar''}\label{kolom-residual-sugar}}

    \begin{tcolorbox}[breakable, size=fbox, boxrule=1pt, pad at break*=1mm,colback=cellbackground, colframe=cellborder]
\prompt{In}{incolor}{18}{\boxspacing}
\begin{Verbatim}[commandchars=\\\{\}]
\PY{n}{var} \PY{o}{=} \PY{n}{df}\PY{p}{[}\PY{l+s+s1}{\PYZsq{}}\PY{l+s+s1}{residual sugar}\PY{l+s+s1}{\PYZsq{}}\PY{p}{]}
\PY{n}{plt}\PY{o}{.}\PY{n}{figure}\PY{p}{(}\PY{n}{figsize}\PY{o}{=}\PY{p}{(}\PY{l+m+mi}{7}\PY{p}{,}\PY{l+m+mi}{6}\PY{p}{)}\PY{p}{)}
\PY{n}{fig}\PY{p}{,}\PY{n}{ax} \PY{o}{=} \PY{n}{plt}\PY{o}{.}\PY{n}{subplots}\PY{p}{(}\PY{l+m+mi}{1}\PY{p}{,}\PY{l+m+mi}{2}\PY{p}{)}
\PY{c+c1}{\PYZsh{} df[var].hist(ax = ax[0]) matplotlib}
\PY{n}{sns}\PY{o}{.}\PY{n}{histplot}\PY{p}{(}\PY{n}{var}\PY{p}{,} \PY{n}{ax}\PY{o}{=}\PY{n}{ax}\PY{p}{[}\PY{l+m+mi}{0}\PY{p}{]}\PY{p}{)}
\PY{n}{ax}\PY{p}{[}\PY{l+m+mi}{0}\PY{p}{]}\PY{o}{.}\PY{n}{set\PYZus{}title}\PY{p}{(}\PY{l+s+s2}{\PYZdq{}}\PY{l+s+s2}{Histogram residual sugar}\PY{l+s+s2}{\PYZdq{}}\PY{p}{)}
\PY{n}{ax}\PY{p}{[}\PY{l+m+mi}{1}\PY{p}{]}\PY{o}{.}\PY{n}{set\PYZus{}title}\PY{p}{(}\PY{l+s+s2}{\PYZdq{}}\PY{l+s+s2}{Boxplot residual sugar}\PY{l+s+s2}{\PYZdq{}}\PY{p}{)}
\PY{c+c1}{\PYZsh{} df.boxplot(var,ax=ax[1]) matplotlib}
\PY{n}{sns}\PY{o}{.}\PY{n}{boxplot}\PY{p}{(}\PY{n}{var}\PY{p}{,}\PY{n}{ax}\PY{o}{=}\PY{n}{ax}\PY{p}{[}\PY{l+m+mi}{1}\PY{p}{]}\PY{p}{)}
\PY{n}{plt}\PY{o}{.}\PY{n}{show}\PY{p}{(}\PY{p}{)}
\end{Verbatim}
\end{tcolorbox}

    
    \begin{Verbatim}[commandchars=\\\{\}]
<Figure size 700x600 with 0 Axes>
    \end{Verbatim}

    
    \begin{center}
    \adjustimage{max size={0.9\linewidth}{0.9\paperheight}}{output_43_1.png}
    \end{center}
    { \hspace*{\fill} \\}
    
    Dari histogram tersebut dapat dilihat bahwa persebaran data untuk kolom
residual sugar \textbf{terdistribusi normal}, apabila dilihat dari
grafik yang berbentuk simetris seperti lonceng dan titik tertinggi
berada di bagian tengah dari data dan rata-rata data yang ada di angka
2.57. Distribusi normal juga dapat dilihat dari kurva bagian samping
yang memiliki frekuensi lebih sedikit dibanding dengan bagian tengah.

Grafik Box Plot juga menunjukkan data yang \textbf{terdistribusi normal}
yang ditunjukkan dari \emph{interquartile range} yang berada di tengah
nilai minimum dan maksimum. Dari box plot juga dapat dilihat terdapat
beberapa data outlier yaitu data yang berada diluar batas atas dan batas
bawah. Terdapat beberapa data outlier yang memiliki nilai lebih dari
batas atas.

    \hypertarget{kolom-chlorides}{%
\subsection{Kolom ``chlorides''}\label{kolom-chlorides}}

    \begin{tcolorbox}[breakable, size=fbox, boxrule=1pt, pad at break*=1mm,colback=cellbackground, colframe=cellborder]
\prompt{In}{incolor}{19}{\boxspacing}
\begin{Verbatim}[commandchars=\\\{\}]
\PY{n}{var} \PY{o}{=} \PY{n}{df}\PY{p}{[}\PY{l+s+s1}{\PYZsq{}}\PY{l+s+s1}{chlorides}\PY{l+s+s1}{\PYZsq{}}\PY{p}{]}
\PY{n}{plt}\PY{o}{.}\PY{n}{figure}\PY{p}{(}\PY{n}{figsize}\PY{o}{=}\PY{p}{(}\PY{l+m+mi}{7}\PY{p}{,}\PY{l+m+mi}{6}\PY{p}{)}\PY{p}{)}
\PY{n}{fig}\PY{p}{,}\PY{n}{ax} \PY{o}{=} \PY{n}{plt}\PY{o}{.}\PY{n}{subplots}\PY{p}{(}\PY{l+m+mi}{1}\PY{p}{,}\PY{l+m+mi}{2}\PY{p}{)}
\PY{c+c1}{\PYZsh{} df[var].hist(ax = ax[0]) matplotlib}
\PY{n}{sns}\PY{o}{.}\PY{n}{histplot}\PY{p}{(}\PY{n}{var}\PY{p}{,} \PY{n}{ax}\PY{o}{=}\PY{n}{ax}\PY{p}{[}\PY{l+m+mi}{0}\PY{p}{]}\PY{p}{)}
\PY{n}{ax}\PY{p}{[}\PY{l+m+mi}{0}\PY{p}{]}\PY{o}{.}\PY{n}{set\PYZus{}title}\PY{p}{(}\PY{l+s+s2}{\PYZdq{}}\PY{l+s+s2}{Histogram chlorides}\PY{l+s+s2}{\PYZdq{}}\PY{p}{)}
\PY{n}{ax}\PY{p}{[}\PY{l+m+mi}{1}\PY{p}{]}\PY{o}{.}\PY{n}{set\PYZus{}title}\PY{p}{(}\PY{l+s+s2}{\PYZdq{}}\PY{l+s+s2}{Boxplot chlorides}\PY{l+s+s2}{\PYZdq{}}\PY{p}{)}
\PY{c+c1}{\PYZsh{} df.boxplot(var,ax=ax[1]) matplotlib}
\PY{n}{sns}\PY{o}{.}\PY{n}{boxplot}\PY{p}{(}\PY{n}{var}\PY{p}{,}\PY{n}{ax}\PY{o}{=}\PY{n}{ax}\PY{p}{[}\PY{l+m+mi}{1}\PY{p}{]}\PY{p}{)}
\PY{n}{plt}\PY{o}{.}\PY{n}{show}\PY{p}{(}\PY{p}{)}

\PY{c+c1}{\PYZsh{} \PYZsh{} plot histogram}
\PY{c+c1}{\PYZsh{} sns.histplot(var, kde=True)}
\PY{c+c1}{\PYZsh{} plt.show()}

\PY{c+c1}{\PYZsh{} plot QQ plot}
\PY{c+c1}{\PYZsh{} probplot(var, dist=\PYZdq{}norm\PYZdq{}, plot=plt)}
\PY{c+c1}{\PYZsh{} plt.show()}
\end{Verbatim}
\end{tcolorbox}

    
    \begin{Verbatim}[commandchars=\\\{\}]
<Figure size 700x600 with 0 Axes>
    \end{Verbatim}

    
    \begin{center}
    \adjustimage{max size={0.9\linewidth}{0.9\paperheight}}{output_46_1.png}
    \end{center}
    { \hspace*{\fill} \\}
    
    Dari histogram tersebut dapat dilihat bahwa persebaran data untuk kolom
chlorides \textbf{terdistribusi normal}, apabila dilihat dari grafik
yang berbentuk simetris seperti lonceng dan titik tertinggi berada di
bagian tengah dari data dan rata-rata data yang ada di angka 0.08.
Distribusi normal juga dapat dilihat dari kurva bagian samping yang
memiliki frekuensi lebih sedikit dibanding dengan bagian tengah.

Grafik Box Plot juga menunjukkan data yang \textbf{terdistribusi normal}
yang ditunjukkan dari \emph{interquartile range} yang berada di tengah
nilai minimum dan maksimum. Dari box plot juga dapat dilihat terdapat
beberapa data outlier yaitu data yang berada diluar batas atas dan batas
bawah. Terdapat beberapa data outlier yang memiliki nilai lebih dari
batas atas yaitu 1 data dan terdapat pula 2 data kurang dari batas
bawah.

    \hypertarget{kolom-free-sulfur-dioxide}{%
\subsection{Kolom ``free sulfur
dioxide''}\label{kolom-free-sulfur-dioxide}}

    \begin{tcolorbox}[breakable, size=fbox, boxrule=1pt, pad at break*=1mm,colback=cellbackground, colframe=cellborder]
\prompt{In}{incolor}{20}{\boxspacing}
\begin{Verbatim}[commandchars=\\\{\}]
\PY{n}{var} \PY{o}{=} \PY{n}{df}\PY{p}{[}\PY{l+s+s1}{\PYZsq{}}\PY{l+s+s1}{free sulfur dioxide}\PY{l+s+s1}{\PYZsq{}}\PY{p}{]}
\PY{n}{plt}\PY{o}{.}\PY{n}{figure}\PY{p}{(}\PY{n}{figsize}\PY{o}{=}\PY{p}{(}\PY{l+m+mi}{7}\PY{p}{,}\PY{l+m+mi}{6}\PY{p}{)}\PY{p}{)}
\PY{n}{fig}\PY{p}{,}\PY{n}{ax} \PY{o}{=} \PY{n}{plt}\PY{o}{.}\PY{n}{subplots}\PY{p}{(}\PY{l+m+mi}{1}\PY{p}{,}\PY{l+m+mi}{2}\PY{p}{)}
\PY{c+c1}{\PYZsh{} df[var].hist(ax = ax[0]) matplotlib}
\PY{n}{sns}\PY{o}{.}\PY{n}{histplot}\PY{p}{(}\PY{n}{var}\PY{p}{,} \PY{n}{ax}\PY{o}{=}\PY{n}{ax}\PY{p}{[}\PY{l+m+mi}{0}\PY{p}{]}\PY{p}{)}
\PY{n}{ax}\PY{p}{[}\PY{l+m+mi}{0}\PY{p}{]}\PY{o}{.}\PY{n}{set\PYZus{}title}\PY{p}{(}\PY{l+s+s2}{\PYZdq{}}\PY{l+s+s2}{Histogram free sulfur dioxide}\PY{l+s+s2}{\PYZdq{}}\PY{p}{)}
\PY{n}{ax}\PY{p}{[}\PY{l+m+mi}{1}\PY{p}{]}\PY{o}{.}\PY{n}{set\PYZus{}title}\PY{p}{(}\PY{l+s+s2}{\PYZdq{}}\PY{l+s+s2}{Boxplot free sulfur dioxide}\PY{l+s+s2}{\PYZdq{}}\PY{p}{)}
\PY{c+c1}{\PYZsh{} df.boxplot(var,ax=ax[1]) matplotlib}
\PY{n}{sns}\PY{o}{.}\PY{n}{boxplot}\PY{p}{(}\PY{n}{var}\PY{p}{,}\PY{n}{ax}\PY{o}{=}\PY{n}{ax}\PY{p}{[}\PY{l+m+mi}{1}\PY{p}{]}\PY{p}{)}
\PY{n}{plt}\PY{o}{.}\PY{n}{show}\PY{p}{(}\PY{p}{)}

\PY{c+c1}{\PYZsh{} probplot(var, dist=\PYZdq{}norm\PYZdq{}, plot=plt)}
\PY{c+c1}{\PYZsh{} plt.show()}
\end{Verbatim}
\end{tcolorbox}

    
    \begin{Verbatim}[commandchars=\\\{\}]
<Figure size 700x600 with 0 Axes>
    \end{Verbatim}

    
    \begin{center}
    \adjustimage{max size={0.9\linewidth}{0.9\paperheight}}{output_49_1.png}
    \end{center}
    { \hspace*{\fill} \\}
    
    Grafik Box Plot juga menunjukkan data yang cukup merata yang ditunjukkan
dari \emph{interquartile range} yang berada di tengah nilai minimum dan
maksimum. Dari box plot juga dapat dilihat terdapat beberapa data
outlier yaitu data yang berada diluar batas atas dan batas bawah.
Terdapat beberapa data outlier yang memiliki nilai kurang dari batas
bawah.

Dari histogram tersebut dapat dilihat bahwa persebaran data untuk kolom
free sulfur dioxide memiliki persebaran paling banyak pada nilai di
sekitar 14.9. Grafik Histogram terlihat cukup simetris dengan persebaran
yang cukup merata.

Grafik Box Plot juga menunjukkan data outlier yaitu data yang berada
diluar batas atas dan batas bawah. Terdapat 6 data yang berada di bawah
batas bawah dan 1 data berada diatas batas atas.

    \hypertarget{kolom-total-sulfur-dioxide}{%
\subsection{Kolom ``total sulfur
dioxide''}\label{kolom-total-sulfur-dioxide}}

    \begin{tcolorbox}[breakable, size=fbox, boxrule=1pt, pad at break*=1mm,colback=cellbackground, colframe=cellborder]
\prompt{In}{incolor}{21}{\boxspacing}
\begin{Verbatim}[commandchars=\\\{\}]
\PY{n}{var} \PY{o}{=} \PY{n}{df}\PY{p}{[}\PY{l+s+s1}{\PYZsq{}}\PY{l+s+s1}{total sulfur dioxide}\PY{l+s+s1}{\PYZsq{}}\PY{p}{]}
\PY{n}{plt}\PY{o}{.}\PY{n}{figure}\PY{p}{(}\PY{n}{figsize}\PY{o}{=}\PY{p}{(}\PY{l+m+mi}{7}\PY{p}{,}\PY{l+m+mi}{6}\PY{p}{)}\PY{p}{)}
\PY{n}{fig}\PY{p}{,}\PY{n}{ax} \PY{o}{=} \PY{n}{plt}\PY{o}{.}\PY{n}{subplots}\PY{p}{(}\PY{l+m+mi}{1}\PY{p}{,}\PY{l+m+mi}{2}\PY{p}{)}
\PY{c+c1}{\PYZsh{} df[var].hist(ax = ax[0]) matplotlib}
\PY{n}{sns}\PY{o}{.}\PY{n}{histplot}\PY{p}{(}\PY{n}{var}\PY{p}{,} \PY{n}{ax}\PY{o}{=}\PY{n}{ax}\PY{p}{[}\PY{l+m+mi}{0}\PY{p}{]}\PY{p}{)}
\PY{n}{ax}\PY{p}{[}\PY{l+m+mi}{0}\PY{p}{]}\PY{o}{.}\PY{n}{set\PYZus{}title}\PY{p}{(}\PY{l+s+s2}{\PYZdq{}}\PY{l+s+s2}{Histogram total sulfur dioxide}\PY{l+s+s2}{\PYZdq{}}\PY{p}{)}
\PY{n}{ax}\PY{p}{[}\PY{l+m+mi}{1}\PY{p}{]}\PY{o}{.}\PY{n}{set\PYZus{}title}\PY{p}{(}\PY{l+s+s2}{\PYZdq{}}\PY{l+s+s2}{Boxplot total sulfur dioxide}\PY{l+s+s2}{\PYZdq{}}\PY{p}{)}
\PY{c+c1}{\PYZsh{} df.boxplot(var,ax=ax[1]) matplotlib}
\PY{n}{sns}\PY{o}{.}\PY{n}{boxplot}\PY{p}{(}\PY{n}{var}\PY{p}{,}\PY{n}{ax}\PY{o}{=}\PY{n}{ax}\PY{p}{[}\PY{l+m+mi}{1}\PY{p}{]}\PY{p}{)}
\PY{n}{plt}\PY{o}{.}\PY{n}{show}\PY{p}{(}\PY{p}{)}
\end{Verbatim}
\end{tcolorbox}

    
    \begin{Verbatim}[commandchars=\\\{\}]
<Figure size 700x600 with 0 Axes>
    \end{Verbatim}

    
    \begin{center}
    \adjustimage{max size={0.9\linewidth}{0.9\paperheight}}{output_52_1.png}
    \end{center}
    { \hspace*{\fill} \\}
    
    Dari histogram tersebut dapat dilihat bahwa persebaran data untuk kolom
total sulfur dioxide \textbf{terdistribusi normal}, apabila dilihat dari
grafik yang berbentuk simetris seperti lonceng dan titik tertinggi
berada di bagian tengah dari data dan rata-rata data yang ada di angka
40.29. Distribusi normal juga dapat dilihat dari kurva bagian samping
yang memiliki frekuensi lebih sedikit dibanding dengan bagian tengah.

Grafik Box Plot juga menunjukkan data yang \textbf{terdistribusi normal}
yang ditunjukkan dari \emph{interquartile range} yang berada di tengah
nilai minimum dan maksimum. Dari box plot juga dapat dilihat terdapat
beberapa data outlier yaitu data yang berada diluar batas atas dan batas
bawah. Terdapat beberapa data outlier yang memiliki nilai kurang dari
batas bawah dan lebih dari batas atas.

    \hypertarget{kolom-density}{%
\subsection{Kolom ``density''}\label{kolom-density}}

    \begin{tcolorbox}[breakable, size=fbox, boxrule=1pt, pad at break*=1mm,colback=cellbackground, colframe=cellborder]
\prompt{In}{incolor}{22}{\boxspacing}
\begin{Verbatim}[commandchars=\\\{\}]
\PY{n}{var} \PY{o}{=} \PY{n}{df}\PY{p}{[}\PY{l+s+s1}{\PYZsq{}}\PY{l+s+s1}{density}\PY{l+s+s1}{\PYZsq{}}\PY{p}{]}
\PY{n}{plt}\PY{o}{.}\PY{n}{figure}\PY{p}{(}\PY{n}{figsize}\PY{o}{=}\PY{p}{(}\PY{l+m+mi}{7}\PY{p}{,}\PY{l+m+mi}{6}\PY{p}{)}\PY{p}{)}
\PY{n}{fig}\PY{p}{,}\PY{n}{ax} \PY{o}{=} \PY{n}{plt}\PY{o}{.}\PY{n}{subplots}\PY{p}{(}\PY{l+m+mi}{1}\PY{p}{,}\PY{l+m+mi}{2}\PY{p}{)}
\PY{c+c1}{\PYZsh{} df[var].hist(ax = ax[0]) matplotlib}
\PY{n}{sns}\PY{o}{.}\PY{n}{histplot}\PY{p}{(}\PY{n}{var}\PY{p}{,} \PY{n}{ax}\PY{o}{=}\PY{n}{ax}\PY{p}{[}\PY{l+m+mi}{0}\PY{p}{]}\PY{p}{)}
\PY{n}{ax}\PY{p}{[}\PY{l+m+mi}{0}\PY{p}{]}\PY{o}{.}\PY{n}{set\PYZus{}title}\PY{p}{(}\PY{l+s+s2}{\PYZdq{}}\PY{l+s+s2}{Histogram density}\PY{l+s+s2}{\PYZdq{}}\PY{p}{)}
\PY{n}{ax}\PY{p}{[}\PY{l+m+mi}{1}\PY{p}{]}\PY{o}{.}\PY{n}{set\PYZus{}title}\PY{p}{(}\PY{l+s+s2}{\PYZdq{}}\PY{l+s+s2}{Boxplot density}\PY{l+s+s2}{\PYZdq{}}\PY{p}{)}
\PY{c+c1}{\PYZsh{} df.boxplot(var,ax=ax[1]) matplotlib}
\PY{n}{sns}\PY{o}{.}\PY{n}{boxplot}\PY{p}{(}\PY{n}{var}\PY{p}{,}\PY{n}{ax}\PY{o}{=}\PY{n}{ax}\PY{p}{[}\PY{l+m+mi}{1}\PY{p}{]}\PY{p}{)}
\PY{n}{plt}\PY{o}{.}\PY{n}{show}\PY{p}{(}\PY{p}{)}
\end{Verbatim}
\end{tcolorbox}

    
    \begin{Verbatim}[commandchars=\\\{\}]
<Figure size 700x600 with 0 Axes>
    \end{Verbatim}

    
    \begin{center}
    \adjustimage{max size={0.9\linewidth}{0.9\paperheight}}{output_55_1.png}
    \end{center}
    { \hspace*{\fill} \\}
    
    Dari histogram tersebut dapat dilihat bahwa persebaran data untuk kolom
density \textbf{terdistribusi normal}, apabila dilihat dari grafik yang
berbentuk simetris seperti lonceng dan titik tertinggi berada di bagian
tengah dari data dan rata-rata data yang ada di angka 0.99. Distribusi
normal juga dapat dilihat dari kurva bagian samping yang memiliki
frekuensi lebih sedikit dibanding dengan bagian tengah.

Grafik Box Plot juga menunjukkan data yang \textbf{terdistribusi normal}
yang ditunjukkan dari \emph{interquartile range} yang berada di tengah
nilai minimum dan maksimum. Dari box plot juga dapat dilihat terdapat
beberapa data outlier yaitu data yang berada diluar batas atas dan batas
bawah. Terdapat beberapa data outlier yang memiliki nilai kurang dari
batas bawah dan lebih dari batas atas.

    \hypertarget{kolom-ph}{%
\subsection{Kolom ``pH''}\label{kolom-ph}}

    \begin{tcolorbox}[breakable, size=fbox, boxrule=1pt, pad at break*=1mm,colback=cellbackground, colframe=cellborder]
\prompt{In}{incolor}{23}{\boxspacing}
\begin{Verbatim}[commandchars=\\\{\}]
\PY{n}{var} \PY{o}{=} \PY{n}{df}\PY{p}{[}\PY{l+s+s1}{\PYZsq{}}\PY{l+s+s1}{pH}\PY{l+s+s1}{\PYZsq{}}\PY{p}{]}
\PY{n}{plt}\PY{o}{.}\PY{n}{figure}\PY{p}{(}\PY{n}{figsize}\PY{o}{=}\PY{p}{(}\PY{l+m+mi}{7}\PY{p}{,}\PY{l+m+mi}{6}\PY{p}{)}\PY{p}{)}
\PY{n}{fig}\PY{p}{,}\PY{n}{ax} \PY{o}{=} \PY{n}{plt}\PY{o}{.}\PY{n}{subplots}\PY{p}{(}\PY{l+m+mi}{1}\PY{p}{,}\PY{l+m+mi}{2}\PY{p}{)}
\PY{c+c1}{\PYZsh{} df[var].hist(ax = ax[0]) matplotlib}
\PY{n}{sns}\PY{o}{.}\PY{n}{histplot}\PY{p}{(}\PY{n}{var}\PY{p}{,} \PY{n}{ax}\PY{o}{=}\PY{n}{ax}\PY{p}{[}\PY{l+m+mi}{0}\PY{p}{]}\PY{p}{)}
\PY{n}{ax}\PY{p}{[}\PY{l+m+mi}{0}\PY{p}{]}\PY{o}{.}\PY{n}{set\PYZus{}title}\PY{p}{(}\PY{l+s+s2}{\PYZdq{}}\PY{l+s+s2}{Histogram pH}\PY{l+s+s2}{\PYZdq{}}\PY{p}{)}
\PY{n}{ax}\PY{p}{[}\PY{l+m+mi}{1}\PY{p}{]}\PY{o}{.}\PY{n}{set\PYZus{}title}\PY{p}{(}\PY{l+s+s2}{\PYZdq{}}\PY{l+s+s2}{Boxplot pH}\PY{l+s+s2}{\PYZdq{}}\PY{p}{)}
\PY{c+c1}{\PYZsh{} df.boxplot(var,ax=ax[1]) matplotlib}
\PY{n}{sns}\PY{o}{.}\PY{n}{boxplot}\PY{p}{(}\PY{n}{var}\PY{p}{,}\PY{n}{ax}\PY{o}{=}\PY{n}{ax}\PY{p}{[}\PY{l+m+mi}{1}\PY{p}{]}\PY{p}{)}
\PY{n}{plt}\PY{o}{.}\PY{n}{show}\PY{p}{(}\PY{p}{)}
\end{Verbatim}
\end{tcolorbox}

    
    \begin{Verbatim}[commandchars=\\\{\}]
<Figure size 700x600 with 0 Axes>
    \end{Verbatim}

    
    \begin{center}
    \adjustimage{max size={0.9\linewidth}{0.9\paperheight}}{output_58_1.png}
    \end{center}
    { \hspace*{\fill} \\}
    
    Dari histogram tersebut dapat dilihat bahwa persebaran data untuk kolom
pH \textbf{terdistribusi normal}, apabila dilihat dari grafik yang
berbentuk simetris seperti lonceng dan titik tertinggi berada di bagian
tengah dari data dan rata-rata data yang ada di angka 3.30. Distribusi
normal juga dapat dilihat dari kurva bagian samping yang memiliki
frekuensi lebih sedikit dibanding dengan bagian tengah.

Grafik Box Plot juga menunjukkan data yang \textbf{terdistribusi normal}
yang ditunjukkan dari \emph{interquartile range} yang berada di tengah
nilai minimum dan maksimum. Dari box plot juga dapat dilihat terdapat
beberapa data outlier yaitu data yang berada diluar batas atas dan batas
bawah. Terdapat 2 data outlier yang memiliki nilai kurang dari batas
bawah dan 3 data yang memiliki nilai lebih dari batas atas.

    \hypertarget{kolom-sulphates}{%
\subsection{Kolom ``sulphates''}\label{kolom-sulphates}}

    \begin{tcolorbox}[breakable, size=fbox, boxrule=1pt, pad at break*=1mm,colback=cellbackground, colframe=cellborder]
\prompt{In}{incolor}{24}{\boxspacing}
\begin{Verbatim}[commandchars=\\\{\}]
\PY{n}{var} \PY{o}{=} \PY{n}{df}\PY{p}{[}\PY{l+s+s1}{\PYZsq{}}\PY{l+s+s1}{sulphates}\PY{l+s+s1}{\PYZsq{}}\PY{p}{]}
\PY{n}{plt}\PY{o}{.}\PY{n}{figure}\PY{p}{(}\PY{n}{figsize}\PY{o}{=}\PY{p}{(}\PY{l+m+mi}{7}\PY{p}{,}\PY{l+m+mi}{6}\PY{p}{)}\PY{p}{)}
\PY{n}{fig}\PY{p}{,}\PY{n}{ax} \PY{o}{=} \PY{n}{plt}\PY{o}{.}\PY{n}{subplots}\PY{p}{(}\PY{l+m+mi}{1}\PY{p}{,}\PY{l+m+mi}{2}\PY{p}{)}
\PY{c+c1}{\PYZsh{} df[var].hist(ax = ax[0]) matplotlib}
\PY{n}{sns}\PY{o}{.}\PY{n}{histplot}\PY{p}{(}\PY{n}{var}\PY{p}{,} \PY{n}{ax}\PY{o}{=}\PY{n}{ax}\PY{p}{[}\PY{l+m+mi}{0}\PY{p}{]}\PY{p}{)}
\PY{n}{ax}\PY{p}{[}\PY{l+m+mi}{0}\PY{p}{]}\PY{o}{.}\PY{n}{set\PYZus{}title}\PY{p}{(}\PY{l+s+s2}{\PYZdq{}}\PY{l+s+s2}{Histogram sulphates}\PY{l+s+s2}{\PYZdq{}}\PY{p}{)}
\PY{n}{ax}\PY{p}{[}\PY{l+m+mi}{1}\PY{p}{]}\PY{o}{.}\PY{n}{set\PYZus{}title}\PY{p}{(}\PY{l+s+s2}{\PYZdq{}}\PY{l+s+s2}{Boxplot sulphates}\PY{l+s+s2}{\PYZdq{}}\PY{p}{)}
\PY{c+c1}{\PYZsh{} df.boxplot(var,ax=ax[1]) matplotlib}
\PY{n}{sns}\PY{o}{.}\PY{n}{boxplot}\PY{p}{(}\PY{n}{var}\PY{p}{,}\PY{n}{ax}\PY{o}{=}\PY{n}{ax}\PY{p}{[}\PY{l+m+mi}{1}\PY{p}{]}\PY{p}{)}
\PY{n}{plt}\PY{o}{.}\PY{n}{show}\PY{p}{(}\PY{p}{)}
\end{Verbatim}
\end{tcolorbox}

    
    \begin{Verbatim}[commandchars=\\\{\}]
<Figure size 700x600 with 0 Axes>
    \end{Verbatim}

    
    \begin{center}
    \adjustimage{max size={0.9\linewidth}{0.9\paperheight}}{output_61_1.png}
    \end{center}
    { \hspace*{\fill} \\}
    
    Dari histogram tersebut dapat dilihat bahwa persebaran data untuk kolom
sulphates \textbf{terdistribusi normal}, apabila dilihat dari grafik
yang berbentuk simetris seperti lonceng dan titik tertinggi berada di
bagian tengah dari data dan rata-rata data yang ada di angka 0.60.
Distribusi normal juga dapat dilihat dari kurva bagian samping yang
memiliki frekuensi lebih sedikit dibanding dengan bagian tengah.

Grafik Box Plot juga menunjukkan data yang \textbf{terdistribusi normal}
yang ditunjukkan dari \emph{interquartile range} yang berada di tengah
nilai minimum dan maksimum. Dari box plot juga dapat dilihat terdapat
beberapa data outlier yaitu data yang berada diluar batas atas dan batas
bawah. Terdapat 1 data outlier yang memiliki nilai kurang dari batas
bawah dan 4 data yang memiliki nilai lebih dari batas atas.

    \hypertarget{kolom-alcohol}{%
\subsection{Kolom ``alcohol''}\label{kolom-alcohol}}

    \begin{tcolorbox}[breakable, size=fbox, boxrule=1pt, pad at break*=1mm,colback=cellbackground, colframe=cellborder]
\prompt{In}{incolor}{25}{\boxspacing}
\begin{Verbatim}[commandchars=\\\{\}]
\PY{n}{var} \PY{o}{=} \PY{n}{df}\PY{p}{[}\PY{l+s+s1}{\PYZsq{}}\PY{l+s+s1}{alcohol}\PY{l+s+s1}{\PYZsq{}}\PY{p}{]}
\PY{n}{plt}\PY{o}{.}\PY{n}{figure}\PY{p}{(}\PY{n}{figsize}\PY{o}{=}\PY{p}{(}\PY{l+m+mi}{7}\PY{p}{,}\PY{l+m+mi}{6}\PY{p}{)}\PY{p}{)}
\PY{n}{fig}\PY{p}{,}\PY{n}{ax} \PY{o}{=} \PY{n}{plt}\PY{o}{.}\PY{n}{subplots}\PY{p}{(}\PY{l+m+mi}{1}\PY{p}{,}\PY{l+m+mi}{2}\PY{p}{)}
\PY{c+c1}{\PYZsh{} df[var].hist(ax = ax[0]) matplotlib}
\PY{n}{sns}\PY{o}{.}\PY{n}{histplot}\PY{p}{(}\PY{n}{var}\PY{p}{,} \PY{n}{ax}\PY{o}{=}\PY{n}{ax}\PY{p}{[}\PY{l+m+mi}{0}\PY{p}{]}\PY{p}{)}
\PY{n}{ax}\PY{p}{[}\PY{l+m+mi}{0}\PY{p}{]}\PY{o}{.}\PY{n}{set\PYZus{}title}\PY{p}{(}\PY{l+s+s2}{\PYZdq{}}\PY{l+s+s2}{Histogram alcohol}\PY{l+s+s2}{\PYZdq{}}\PY{p}{)}
\PY{n}{ax}\PY{p}{[}\PY{l+m+mi}{1}\PY{p}{]}\PY{o}{.}\PY{n}{set\PYZus{}title}\PY{p}{(}\PY{l+s+s2}{\PYZdq{}}\PY{l+s+s2}{Boxplot alcohol}\PY{l+s+s2}{\PYZdq{}}\PY{p}{)}
\PY{c+c1}{\PYZsh{} df.boxplot(var,ax=ax[1]) matplotlib}
\PY{n}{sns}\PY{o}{.}\PY{n}{boxplot}\PY{p}{(}\PY{n}{var}\PY{p}{,}\PY{n}{ax}\PY{o}{=}\PY{n}{ax}\PY{p}{[}\PY{l+m+mi}{1}\PY{p}{]}\PY{p}{)}
\PY{n}{plt}\PY{o}{.}\PY{n}{show}\PY{p}{(}\PY{p}{)}
\end{Verbatim}
\end{tcolorbox}

    
    \begin{Verbatim}[commandchars=\\\{\}]
<Figure size 700x600 with 0 Axes>
    \end{Verbatim}

    
    \begin{center}
    \adjustimage{max size={0.9\linewidth}{0.9\paperheight}}{output_64_1.png}
    \end{center}
    { \hspace*{\fill} \\}
    
    \begin{tcolorbox}[breakable, size=fbox, boxrule=1pt, pad at break*=1mm,colback=cellbackground, colframe=cellborder]
\prompt{In}{incolor}{26}{\boxspacing}
\begin{Verbatim}[commandchars=\\\{\}]
\PY{n}{var}\PY{o}{.}\PY{n}{mean}\PY{p}{(}\PY{p}{)}
\end{Verbatim}
\end{tcolorbox}

            \begin{tcolorbox}[breakable, size=fbox, boxrule=.5pt, pad at break*=1mm, opacityfill=0]
\prompt{Out}{outcolor}{26}{\boxspacing}
\begin{Verbatim}[commandchars=\\\{\}]
10.592279999999999
\end{Verbatim}
\end{tcolorbox}
        
    Dari histogram tersebut dapat dilihat bahwa persebaran data untuk kolom
alcohol \textbf{terdistribusi normal}, apabila dilihat dari grafik yang
berbentuk simetris seperti lonceng dan titik tertinggi berada di bagian
tengah dari data dan rata-rata data yang ada di angka 10.59. Distribusi
normal juga dapat dilihat dari kurva bagian samping yang memiliki
frekuensi lebih sedikit dibanding dengan bagian tengah.

Grafik Box Plot juga menunjukkan data yang \textbf{terdistribusi normal}
yang ditunjukkan dari \emph{interquartile range} yang berada di tengah
nilai minimum dan maksimum. Dari box plot juga dapat dilihat terdapat
beberapa data outlier yaitu data yang berada diluar batas atas dan batas
bawah. Terdapat beberapa data outlier yang memiliki nilai kurang dari
batas bawah dan 1 data yang memiliki nilai lebih dari batas atas.

    \begin{tcolorbox}[breakable, size=fbox, boxrule=1pt, pad at break*=1mm,colback=cellbackground, colframe=cellborder]
\prompt{In}{incolor}{27}{\boxspacing}
\begin{Verbatim}[commandchars=\\\{\}]
\PY{n}{var} \PY{o}{=} \PY{n}{df}\PY{p}{[}\PY{l+s+s1}{\PYZsq{}}\PY{l+s+s1}{quality}\PY{l+s+s1}{\PYZsq{}}\PY{p}{]}
\PY{n}{plt}\PY{o}{.}\PY{n}{figure}\PY{p}{(}\PY{n}{figsize}\PY{o}{=}\PY{p}{(}\PY{l+m+mi}{7}\PY{p}{,}\PY{l+m+mi}{6}\PY{p}{)}\PY{p}{)}
\PY{n}{fig}\PY{p}{,}\PY{n}{ax} \PY{o}{=} \PY{n}{plt}\PY{o}{.}\PY{n}{subplots}\PY{p}{(}\PY{l+m+mi}{1}\PY{p}{,}\PY{l+m+mi}{2}\PY{p}{)}
\PY{c+c1}{\PYZsh{} df[var].hist(ax = ax[0]) matplotlib}
\PY{n}{sns}\PY{o}{.}\PY{n}{histplot}\PY{p}{(}\PY{n}{var}\PY{p}{,} \PY{n}{ax}\PY{o}{=}\PY{n}{ax}\PY{p}{[}\PY{l+m+mi}{0}\PY{p}{]}\PY{p}{)}
\PY{n}{ax}\PY{p}{[}\PY{l+m+mi}{0}\PY{p}{]}\PY{o}{.}\PY{n}{set\PYZus{}title}\PY{p}{(}\PY{l+s+s2}{\PYZdq{}}\PY{l+s+s2}{Histogram quality}\PY{l+s+s2}{\PYZdq{}}\PY{p}{)}
\PY{n}{ax}\PY{p}{[}\PY{l+m+mi}{1}\PY{p}{]}\PY{o}{.}\PY{n}{set\PYZus{}title}\PY{p}{(}\PY{l+s+s2}{\PYZdq{}}\PY{l+s+s2}{Boxplot quality}\PY{l+s+s2}{\PYZdq{}}\PY{p}{)}
\PY{c+c1}{\PYZsh{} df.boxplot(var,ax=ax[1]) matplotlib}
\PY{n}{sns}\PY{o}{.}\PY{n}{boxplot}\PY{p}{(}\PY{n}{var}\PY{p}{,}\PY{n}{ax}\PY{o}{=}\PY{n}{ax}\PY{p}{[}\PY{l+m+mi}{1}\PY{p}{]}\PY{p}{)}
\PY{n}{plt}\PY{o}{.}\PY{n}{show}\PY{p}{(}\PY{p}{)}
\end{Verbatim}
\end{tcolorbox}

    
    \begin{Verbatim}[commandchars=\\\{\}]
<Figure size 700x600 with 0 Axes>
    \end{Verbatim}

    
    \begin{center}
    \adjustimage{max size={0.9\linewidth}{0.9\paperheight}}{output_67_1.png}
    \end{center}
    { \hspace*{\fill} \\}
    
    \hypertarget{soal-3}{%
\section{SOAL 3}\label{soal-3}}

    Dibawah ini terdefinisi prosedur untuk pengecekan suatu data
terdistribusi normal atau tidak. Menggunakan D'Agostino Test dan Pearson
Test yang menggabungkan tes skewness dan kurtosis dengan hasil
\[s^2 + k^2\]

dengan s nilai z dari skewness test dan k nilai z dari kurtosis test

Parameter yang digunakan:

\(H_0\): Data berdistribusi normal

\(H_1\): Data tidak berdistribusi normal

\(\alpha\): 0.05

Jika nilai \(p < \alpha\), maka \(H_0\) ditolak, sebaliknya
\(p \geq \alpha\), maka \(H_0\) diterima

    \begin{tcolorbox}[breakable, size=fbox, boxrule=1pt, pad at break*=1mm,colback=cellbackground, colframe=cellborder]
\prompt{In}{incolor}{28}{\boxspacing}
\begin{Verbatim}[commandchars=\\\{\}]
\PY{k+kn}{from} \PY{n+nn}{scipy}\PY{n+nn}{.}\PY{n+nn}{stats} \PY{k+kn}{import} \PY{n}{normaltest}

\PY{k}{def} \PY{n+nf}{normal\PYZus{}test}\PY{p}{(}\PY{n}{col}\PY{p}{)}\PY{p}{:}
    \PY{n}{stat}\PY{p}{,} \PY{n}{p} \PY{o}{=} \PY{n}{normaltest}\PY{p}{(}\PY{n}{df}\PY{p}{[}\PY{n}{col}\PY{p}{]}\PY{p}{)}

    \PY{k}{if} \PY{n}{p} \PY{o}{\PYZgt{}} \PY{l+m+mf}{0.05}\PY{p}{:}
        \PY{n}{display}\PY{p}{(}\PY{n}{Markdown}\PY{p}{(}\PY{l+s+s2}{\PYZdq{}}\PY{l+s+s2}{Kolom }\PY{l+s+s2}{\PYZdq{}} \PY{o}{+} \PY{n}{col} \PY{o}{+}\PY{l+s+s2}{\PYZdq{}}\PY{l+s+s2}{ **terdistribusi normal**}\PY{l+s+s2}{\PYZdq{}}\PY{p}{)}\PY{p}{)}
    \PY{k}{else}\PY{p}{:}
        \PY{n}{display}\PY{p}{(}\PY{n}{Markdown}\PY{p}{(}\PY{l+s+s2}{\PYZdq{}}\PY{l+s+s2}{Kolom }\PY{l+s+s2}{\PYZdq{}} \PY{o}{+} \PY{n}{col} \PY{o}{+}\PY{l+s+s2}{\PYZdq{}}\PY{l+s+s2}{ **tidak terdistribusi normal**}\PY{l+s+s2}{\PYZdq{}}\PY{p}{)}\PY{p}{)}

    \PY{n}{sns}\PY{o}{.}\PY{n}{histplot}\PY{p}{(}\PY{n}{df}\PY{p}{[}\PY{n}{col}\PY{p}{]}\PY{p}{,} \PY{n}{bins}\PY{o}{=}\PY{l+s+s2}{\PYZdq{}}\PY{l+s+s2}{auto}\PY{l+s+s2}{\PYZdq{}}\PY{p}{,} \PY{n}{kde} \PY{o}{=} \PY{k+kc}{True}\PY{p}{)}
    \PY{n}{plt}\PY{o}{.}\PY{n}{title}\PY{p}{(}\PY{l+s+s1}{\PYZsq{}}\PY{l+s+s1}{Histogram of }\PY{l+s+s1}{\PYZsq{}} \PY{o}{+} \PY{n}{col}\PY{p}{)}
    \PY{n}{plt}\PY{o}{.}\PY{n}{xlabel}\PY{p}{(}\PY{l+s+s1}{\PYZsq{}}\PY{l+s+s1}{Value}\PY{l+s+s1}{\PYZsq{}}\PY{p}{)}
    \PY{n}{plt}\PY{o}{.}\PY{n}{ylabel}\PY{p}{(}\PY{l+s+s1}{\PYZsq{}}\PY{l+s+s1}{Frequency}\PY{l+s+s1}{\PYZsq{}}\PY{p}{)}
    \PY{n}{plt}\PY{o}{.}\PY{n}{show}\PY{p}{(}\PY{p}{)}
\end{Verbatim}
\end{tcolorbox}

    \hypertarget{kolom-fixed-acidity}{%
\subsection{Kolom Fixed Acidity}\label{kolom-fixed-acidity}}

    \begin{tcolorbox}[breakable, size=fbox, boxrule=1pt, pad at break*=1mm,colback=cellbackground, colframe=cellborder]
\prompt{In}{incolor}{29}{\boxspacing}
\begin{Verbatim}[commandchars=\\\{\}]
\PY{n}{normal\PYZus{}test}\PY{p}{(}\PY{l+s+s1}{\PYZsq{}}\PY{l+s+s1}{fixed acidity}\PY{l+s+s1}{\PYZsq{}}\PY{p}{)}
\end{Verbatim}
\end{tcolorbox}

    Kolom fixed acidity \textbf{terdistribusi normal}

    
    \begin{center}
    \adjustimage{max size={0.9\linewidth}{0.9\paperheight}}{output_72_1.png}
    \end{center}
    { \hspace*{\fill} \\}
    
    \hypertarget{kolom-volatile-acidity}{%
\subsection{Kolom Volatile Acidity}\label{kolom-volatile-acidity}}

    \begin{tcolorbox}[breakable, size=fbox, boxrule=1pt, pad at break*=1mm,colback=cellbackground, colframe=cellborder]
\prompt{In}{incolor}{30}{\boxspacing}
\begin{Verbatim}[commandchars=\\\{\}]
\PY{n}{normal\PYZus{}test}\PY{p}{(}\PY{l+s+s1}{\PYZsq{}}\PY{l+s+s1}{volatile acidity}\PY{l+s+s1}{\PYZsq{}}\PY{p}{)}
\end{Verbatim}
\end{tcolorbox}

    Kolom volatile acidity \textbf{tidak terdistribusi normal}

    
    \begin{center}
    \adjustimage{max size={0.9\linewidth}{0.9\paperheight}}{output_74_1.png}
    \end{center}
    { \hspace*{\fill} \\}
    
    \hypertarget{kolom-citric-acid}{%
\subsection{Kolom Citric Acid}\label{kolom-citric-acid}}

    \begin{tcolorbox}[breakable, size=fbox, boxrule=1pt, pad at break*=1mm,colback=cellbackground, colframe=cellborder]
\prompt{In}{incolor}{31}{\boxspacing}
\begin{Verbatim}[commandchars=\\\{\}]
\PY{n}{normal\PYZus{}test}\PY{p}{(}\PY{l+s+s1}{\PYZsq{}}\PY{l+s+s1}{citric acid}\PY{l+s+s1}{\PYZsq{}}\PY{p}{)}
\end{Verbatim}
\end{tcolorbox}

    Kolom citric acid \textbf{terdistribusi normal}

    
    \begin{center}
    \adjustimage{max size={0.9\linewidth}{0.9\paperheight}}{output_76_1.png}
    \end{center}
    { \hspace*{\fill} \\}
    
    \hypertarget{kolom-residual-sugar}{%
\subsection{Kolom Residual Sugar}\label{kolom-residual-sugar}}

    \begin{tcolorbox}[breakable, size=fbox, boxrule=1pt, pad at break*=1mm,colback=cellbackground, colframe=cellborder]
\prompt{In}{incolor}{32}{\boxspacing}
\begin{Verbatim}[commandchars=\\\{\}]
\PY{n}{normal\PYZus{}test}\PY{p}{(}\PY{l+s+s1}{\PYZsq{}}\PY{l+s+s1}{residual sugar}\PY{l+s+s1}{\PYZsq{}}\PY{p}{)}
\end{Verbatim}
\end{tcolorbox}

    Kolom residual sugar \textbf{terdistribusi normal}

    
    \begin{center}
    \adjustimage{max size={0.9\linewidth}{0.9\paperheight}}{output_78_1.png}
    \end{center}
    { \hspace*{\fill} \\}
    
    \hypertarget{kolom-chlorides}{%
\subsection{Kolom Chlorides}\label{kolom-chlorides}}

    \begin{tcolorbox}[breakable, size=fbox, boxrule=1pt, pad at break*=1mm,colback=cellbackground, colframe=cellborder]
\prompt{In}{incolor}{33}{\boxspacing}
\begin{Verbatim}[commandchars=\\\{\}]
\PY{n}{normal\PYZus{}test}\PY{p}{(}\PY{l+s+s1}{\PYZsq{}}\PY{l+s+s1}{chlorides}\PY{l+s+s1}{\PYZsq{}}\PY{p}{)}
\end{Verbatim}
\end{tcolorbox}

    Kolom chlorides \textbf{terdistribusi normal}

    
    \begin{center}
    \adjustimage{max size={0.9\linewidth}{0.9\paperheight}}{output_80_1.png}
    \end{center}
    { \hspace*{\fill} \\}
    
    \hypertarget{kolom-free-sulfur-dioxide}{%
\subsection{Kolom Free Sulfur Dioxide}\label{kolom-free-sulfur-dioxide}}

    \begin{tcolorbox}[breakable, size=fbox, boxrule=1pt, pad at break*=1mm,colback=cellbackground, colframe=cellborder]
\prompt{In}{incolor}{34}{\boxspacing}
\begin{Verbatim}[commandchars=\\\{\}]
\PY{n}{normal\PYZus{}test}\PY{p}{(}\PY{l+s+s1}{\PYZsq{}}\PY{l+s+s1}{free sulfur dioxide}\PY{l+s+s1}{\PYZsq{}}\PY{p}{)}
\end{Verbatim}
\end{tcolorbox}

    Kolom free sulfur dioxide \textbf{tidak terdistribusi normal}

    
    \begin{center}
    \adjustimage{max size={0.9\linewidth}{0.9\paperheight}}{output_82_1.png}
    \end{center}
    { \hspace*{\fill} \\}
    
    \hypertarget{kolom-total-sulfur-dioxide}{%
\subsection{Kolom Total Sulfur
Dioxide}\label{kolom-total-sulfur-dioxide}}

    \begin{tcolorbox}[breakable, size=fbox, boxrule=1pt, pad at break*=1mm,colback=cellbackground, colframe=cellborder]
\prompt{In}{incolor}{35}{\boxspacing}
\begin{Verbatim}[commandchars=\\\{\}]
\PY{n}{normal\PYZus{}test}\PY{p}{(}\PY{l+s+s1}{\PYZsq{}}\PY{l+s+s1}{total sulfur dioxide}\PY{l+s+s1}{\PYZsq{}}\PY{p}{)}
\end{Verbatim}
\end{tcolorbox}

    Kolom total sulfur dioxide \textbf{terdistribusi normal}

    
    \begin{center}
    \adjustimage{max size={0.9\linewidth}{0.9\paperheight}}{output_84_1.png}
    \end{center}
    { \hspace*{\fill} \\}
    
    \hypertarget{kolom-density}{%
\subsection{Kolom Density}\label{kolom-density}}

    \begin{tcolorbox}[breakable, size=fbox, boxrule=1pt, pad at break*=1mm,colback=cellbackground, colframe=cellborder]
\prompt{In}{incolor}{36}{\boxspacing}
\begin{Verbatim}[commandchars=\\\{\}]
\PY{n}{normal\PYZus{}test}\PY{p}{(}\PY{l+s+s1}{\PYZsq{}}\PY{l+s+s1}{density}\PY{l+s+s1}{\PYZsq{}}\PY{p}{)}
\end{Verbatim}
\end{tcolorbox}

    Kolom density \textbf{terdistribusi normal}

    
    \begin{center}
    \adjustimage{max size={0.9\linewidth}{0.9\paperheight}}{output_86_1.png}
    \end{center}
    { \hspace*{\fill} \\}
    
    \hypertarget{kolom-ph}{%
\subsection{Kolom pH}\label{kolom-ph}}

    \begin{tcolorbox}[breakable, size=fbox, boxrule=1pt, pad at break*=1mm,colback=cellbackground, colframe=cellborder]
\prompt{In}{incolor}{37}{\boxspacing}
\begin{Verbatim}[commandchars=\\\{\}]
\PY{n}{normal\PYZus{}test}\PY{p}{(}\PY{l+s+s1}{\PYZsq{}}\PY{l+s+s1}{pH}\PY{l+s+s1}{\PYZsq{}}\PY{p}{)}
\end{Verbatim}
\end{tcolorbox}

    Kolom pH \textbf{terdistribusi normal}

    
    \begin{center}
    \adjustimage{max size={0.9\linewidth}{0.9\paperheight}}{output_88_1.png}
    \end{center}
    { \hspace*{\fill} \\}
    
    \hypertarget{kolom-sulphates}{%
\subsection{Kolom Sulphates}\label{kolom-sulphates}}

    \begin{tcolorbox}[breakable, size=fbox, boxrule=1pt, pad at break*=1mm,colback=cellbackground, colframe=cellborder]
\prompt{In}{incolor}{38}{\boxspacing}
\begin{Verbatim}[commandchars=\\\{\}]
\PY{n}{normal\PYZus{}test}\PY{p}{(}\PY{l+s+s1}{\PYZsq{}}\PY{l+s+s1}{sulphates}\PY{l+s+s1}{\PYZsq{}}\PY{p}{)}
\end{Verbatim}
\end{tcolorbox}

    Kolom sulphates \textbf{terdistribusi normal}

    
    \begin{center}
    \adjustimage{max size={0.9\linewidth}{0.9\paperheight}}{output_90_1.png}
    \end{center}
    { \hspace*{\fill} \\}
    
    \hypertarget{kolom-alcohol}{%
\subsection{Kolom Alcohol}\label{kolom-alcohol}}

    \begin{tcolorbox}[breakable, size=fbox, boxrule=1pt, pad at break*=1mm,colback=cellbackground, colframe=cellborder]
\prompt{In}{incolor}{39}{\boxspacing}
\begin{Verbatim}[commandchars=\\\{\}]
\PY{n}{normal\PYZus{}test}\PY{p}{(}\PY{l+s+s1}{\PYZsq{}}\PY{l+s+s1}{alcohol}\PY{l+s+s1}{\PYZsq{}}\PY{p}{)}
\end{Verbatim}
\end{tcolorbox}

    Kolom alcohol \textbf{terdistribusi normal}

    
    \begin{center}
    \adjustimage{max size={0.9\linewidth}{0.9\paperheight}}{output_92_1.png}
    \end{center}
    { \hspace*{\fill} \\}
    
    \hypertarget{soal-4}{%
\section{SOAL 4}\label{soal-4}}

    \hypertarget{a.-nilai-rata-rata-ph-di-atas-3.29}{%
\subsection{a. Nilai rata-rata pH di atas
3.29?}\label{a.-nilai-rata-rata-ph-di-atas-3.29}}

    \textbf{Tentukan Hipotesis null}

\(H_0 : \mu = 3.29\)

\textbf{Tentukan Hipotesis alternatif}

\(H_1 : \mu > 3.29\)

\textbf{Tentukan tingkat signifikan}

\(\alpha = 0.05\)

\textbf{Menggunakan one-tailed test dan perhitungan z}

Nilai z didapatkan dari
\[ z_{Hit} = \frac{\bar{x}-\mu}{\sigma}  \sqrt{N}\]

Daerah kritis yang diambil = \(z > 1.645\)

\textbf{Perhitungan \(p\)-value}

\(p\)-value didapatkan dengan menghitung \(P(Z > z_{Hit})\)

\textbf{Pengambilan Keputusan}

\(H_0\) diterima jika \(p \geq \alpha\) dan \(z_{Hit} \leq z_{Tab}\)

\(H_0\) ditolak jika \(p < \alpha\) atau \(z_{Hit} > z_{Tab}\)

    \begin{tcolorbox}[breakable, size=fbox, boxrule=1pt, pad at break*=1mm,colback=cellbackground, colframe=cellborder]
\prompt{In}{incolor}{40}{\boxspacing}
\begin{Verbatim}[commandchars=\\\{\}]
\PY{c+c1}{\PYZsh{} Nilai Kepercayaan}
\PY{n}{sig} \PY{o}{=} \PY{l+m+mf}{0.05}

\PY{c+c1}{\PYZsh{} cari nilai z}
\PY{n}{zHit} \PY{o}{=} \PY{p}{(}\PY{n}{df}\PY{p}{[}\PY{l+s+s1}{\PYZsq{}}\PY{l+s+s1}{pH}\PY{l+s+s1}{\PYZsq{}}\PY{p}{]}\PY{o}{.}\PY{n}{mean}\PY{p}{(}\PY{p}{)} \PY{o}{\PYZhy{}} \PY{l+m+mf}{3.29}\PY{p}{)}\PY{o}{/}\PY{p}{(}\PY{n}{df}\PY{p}{[}\PY{l+s+s1}{\PYZsq{}}\PY{l+s+s1}{pH}\PY{l+s+s1}{\PYZsq{}}\PY{p}{]}\PY{o}{.}\PY{n}{std}\PY{p}{(}\PY{p}{)}\PY{o}{/}\PY{n}{np}\PY{o}{.}\PY{n}{sqrt}\PY{p}{(}\PY{l+m+mi}{1000}\PY{p}{)}\PY{p}{)}

\PY{c+c1}{\PYZsh{} cari nilai z dari tabel}
\PY{n}{zTab} \PY{o}{=} \PY{n}{norm}\PY{o}{.}\PY{n}{ppf}\PY{p}{(}\PY{l+m+mi}{1}\PY{o}{\PYZhy{}}\PY{n}{sig}\PY{p}{)}

\PY{c+c1}{\PYZsh{}cari Pvalue}
\PY{n}{Pval} \PY{o}{=} \PY{n}{norm}\PY{o}{.}\PY{n}{sf}\PY{p}{(}\PY{n}{zHit}\PY{p}{)}

\PY{n}{display}\PY{p}{(}\PY{n}{Markdown}\PY{p}{(}\PY{l+s+sa}{f}\PY{l+s+s2}{\PYZdq{}}\PY{l+s+s2}{Nilai p = }\PY{l+s+si}{\PYZob{}}\PY{n+nb}{round}\PY{p}{(}\PY{n}{Pval}\PY{p}{,}\PY{l+m+mi}{3}\PY{p}{)}\PY{l+s+si}{\PYZcb{}}\PY{l+s+s2}{\PYZdq{}}\PY{p}{)}\PY{p}{)}
\PY{n}{display}\PY{p}{(}\PY{n}{Markdown}\PY{p}{(}\PY{l+s+sa}{f}\PY{l+s+s2}{\PYZdq{}}\PY{l+s+s2}{Nilai z = }\PY{l+s+si}{\PYZob{}}\PY{n+nb}{round}\PY{p}{(}\PY{n}{zHit}\PY{p}{,}\PY{l+m+mi}{3}\PY{p}{)}\PY{l+s+si}{\PYZcb{}}\PY{l+s+s2}{\PYZdq{}}\PY{p}{)}\PY{p}{)}
\PY{n}{display}\PY{p}{(}\PY{n}{Markdown}\PY{p}{(}\PY{l+s+sa}{f}\PY{l+s+s2}{\PYZdq{}}\PY{l+s+s2}{Nilai alpha = }\PY{l+s+si}{\PYZob{}}\PY{n+nb}{round}\PY{p}{(}\PY{n}{sig}\PY{p}{,}\PY{l+m+mi}{3}\PY{p}{)}\PY{l+s+si}{\PYZcb{}}\PY{l+s+s2}{\PYZdq{}}\PY{p}{)}\PY{p}{)}
\PY{n}{display}\PY{p}{(}\PY{n}{Markdown}\PY{p}{(}\PY{l+s+sa}{f}\PY{l+s+s2}{\PYZdq{}}\PY{l+s+s2}{Nilai z tabel = }\PY{l+s+si}{\PYZob{}}\PY{n+nb}{round}\PY{p}{(}\PY{n}{zTab}\PY{p}{,}\PY{l+m+mi}{3}\PY{p}{)}\PY{l+s+si}{\PYZcb{}}\PY{l+s+s2}{\PYZdq{}}\PY{p}{)}\PY{p}{)}

\PY{n}{display}\PY{p}{(}\PY{n}{Markdown}\PY{p}{(}\PY{l+s+s2}{\PYZdq{}}\PY{l+s+s2}{Kesimpulan}\PY{l+s+s2}{\PYZdq{}}\PY{p}{)}\PY{p}{)}
\PY{k}{if} \PY{p}{(}\PY{n}{Pval} \PY{o}{\PYZgt{}}\PY{o}{=} \PY{n}{sig} \PY{o+ow}{and} \PY{n}{zHit} \PY{o}{\PYZlt{}}\PY{o}{=} \PY{n}{zTab}\PY{p}{)}\PY{p}{:}
    \PY{n}{display}\PY{p}{(}\PY{n}{Markdown}\PY{p}{(}\PY{l+s+s2}{\PYZdq{}}\PY{l+s+s2}{H0 diterima: Rata\PYZhy{}rata pH\PYZhy{}nya 3.29 ke bawah}\PY{l+s+s2}{\PYZdq{}}\PY{p}{)}\PY{p}{)}
\PY{k}{else}\PY{p}{:}
    \PY{n}{display}\PY{p}{(}\PY{n}{Markdown}\PY{p}{(}\PY{l+s+s2}{\PYZdq{}}\PY{l+s+s2}{H0 ditolak: Rata\PYZhy{}rata pH\PYZhy{}nya lebih dari 3.29}\PY{l+s+s2}{\PYZdq{}}\PY{p}{)}\PY{p}{)}
\end{Verbatim}
\end{tcolorbox}

    Nilai p = 0.0

    
    Nilai z = 4.104

    
    Nilai alpha = 0.05

    
    Nilai z tabel = 1.645

    
    Kesimpulan

    
    H0 ditolak: Rata-rata pH-nya lebih dari 3.29

    
    \hypertarget{b.-nilai-rata-rata-residual-sugar-tidak-sama-dengan-2.50}{%
\subsection{b. Nilai rata-rata Residual Sugar tidak sama dengan
2.50?}\label{b.-nilai-rata-rata-residual-sugar-tidak-sama-dengan-2.50}}

    \textbf{Tentukan Hipotesis null}

\(H_0 : \mu = 2.5\)

\textbf{Tentukan Hipotesis alternatif}

\(H_1 : \mu \neq 2.5\)

\textbf{Tentukan tingkat signifikan}

\(\alpha = 0.05\)

\textbf{Menggunakan two-tailed test dan perhitungan z}

Nilai z didapatkan dari
\[ z_{Hit} = \frac{\bar{x}-\mu}{\sigma}  \sqrt{N}\]

daerah kritis yang diambil \(z < -1.96\) atau \(z > 1.96\)

\textbf{Perhitungan \(p\)-value}

\(p\)-value didapatkan dari \(2P(Z<-z_{Hit})\)

\textbf{Pengambilan Keputusan}

\(H_0\) ditolak jika \(p < \alpha\) dan untuk nilai \(z\), \(z < -1.96\)
atau \(z > 1.96\)

\(H_0\) diterima jika \(p \geq \alpha\) dan untuk nilai \(z\),
\(-1.96 \leq z \leq 1.96\)

    \begin{tcolorbox}[breakable, size=fbox, boxrule=1pt, pad at break*=1mm,colback=cellbackground, colframe=cellborder]
\prompt{In}{incolor}{41}{\boxspacing}
\begin{Verbatim}[commandchars=\\\{\}]
\PY{c+c1}{\PYZsh{}4.b}
\PY{c+c1}{\PYZsh{} Nilai Kepercayaan = 0.05}
\PY{n}{sig} \PY{o}{=} \PY{l+m+mf}{0.05}
\PY{c+c1}{\PYZsh{} Hitung nilai ZHit}
\PY{n}{zHit} \PY{o}{=} \PY{p}{(}\PY{n}{df}\PY{p}{[}\PY{l+s+s1}{\PYZsq{}}\PY{l+s+s1}{residual sugar}\PY{l+s+s1}{\PYZsq{}}\PY{p}{]}\PY{o}{.}\PY{n}{mean}\PY{p}{(}\PY{p}{)} \PY{o}{\PYZhy{}} \PY{l+m+mf}{2.5}\PY{p}{)}\PY{o}{/}\PY{p}{(}\PY{n}{df}\PY{p}{[}\PY{l+s+s1}{\PYZsq{}}\PY{l+s+s1}{residual sugar}\PY{l+s+s1}{\PYZsq{}}\PY{p}{]}\PY{o}{.}\PY{n}{std}\PY{p}{(}\PY{p}{)}\PY{o}{/}\PY{n}{np}\PY{o}{.}\PY{n}{sqrt}\PY{p}{(}\PY{l+m+mi}{1000}\PY{p}{)}\PY{p}{)}
\PY{c+c1}{\PYZsh{} Cari nilai Z Tabel}
\PY{n}{zTab} \PY{o}{=} \PY{n}{norm}\PY{o}{.}\PY{n}{ppf}\PY{p}{(}\PY{l+m+mi}{1} \PY{o}{\PYZhy{}} \PY{n}{sig}\PY{o}{/}\PY{l+m+mi}{2}\PY{p}{)}

\PY{c+c1}{\PYZsh{}cari p value untuk two tailed}
\PY{n}{Pval} \PY{o}{=} \PY{l+m+mi}{2}\PY{o}{*}\PY{n}{norm}\PY{o}{.}\PY{n}{sf}\PY{p}{(}\PY{n}{zHit}\PY{p}{)}

\PY{n}{display}\PY{p}{(}\PY{n}{Markdown}\PY{p}{(}\PY{l+s+sa}{f}\PY{l+s+s2}{\PYZdq{}}\PY{l+s+s2}{Nilai p = }\PY{l+s+si}{\PYZob{}}\PY{n+nb}{round}\PY{p}{(}\PY{n}{Pval}\PY{p}{,}\PY{l+m+mi}{3}\PY{p}{)}\PY{l+s+si}{\PYZcb{}}\PY{l+s+s2}{\PYZdq{}}\PY{p}{)}\PY{p}{)}
\PY{n}{display}\PY{p}{(}\PY{n}{Markdown}\PY{p}{(}\PY{l+s+sa}{f}\PY{l+s+s2}{\PYZdq{}}\PY{l+s+s2}{Nilai z = }\PY{l+s+si}{\PYZob{}}\PY{n+nb}{round}\PY{p}{(}\PY{n}{zHit}\PY{p}{,}\PY{l+m+mi}{3}\PY{p}{)}\PY{l+s+si}{\PYZcb{}}\PY{l+s+s2}{\PYZdq{}}\PY{p}{)}\PY{p}{)}
\PY{n}{display}\PY{p}{(}\PY{n}{Markdown}\PY{p}{(}\PY{l+s+sa}{f}\PY{l+s+s2}{\PYZdq{}}\PY{l+s+s2}{Nilai alpha = }\PY{l+s+si}{\PYZob{}}\PY{n+nb}{round}\PY{p}{(}\PY{n}{sig}\PY{p}{,}\PY{l+m+mi}{3}\PY{p}{)}\PY{l+s+si}{\PYZcb{}}\PY{l+s+s2}{\PYZdq{}}\PY{p}{)}\PY{p}{)}
\PY{n}{display}\PY{p}{(}\PY{n}{Markdown}\PY{p}{(}\PY{l+s+sa}{f}\PY{l+s+s2}{\PYZdq{}}\PY{l+s+s2}{Nilai z tabel = }\PY{l+s+si}{\PYZob{}}\PY{n+nb}{round}\PY{p}{(}\PY{n}{zTab}\PY{p}{,}\PY{l+m+mi}{3}\PY{p}{)}\PY{l+s+si}{\PYZcb{}}\PY{l+s+s2}{\PYZdq{}}\PY{p}{)}\PY{p}{)}

\PY{c+c1}{\PYZsh{} Pengambilan Keputusan}
\PY{k}{if} \PY{p}{(}\PY{n}{Pval}\PY{o}{\PYZlt{}}\PY{n}{sig} \PY{o+ow}{and} \PY{p}{(}\PY{n}{zHit} \PY{o}{\PYZlt{}} \PY{o}{\PYZhy{}}\PY{l+m+mi}{1}\PY{o}{*}\PY{p}{(}\PY{n}{zTab}\PY{p}{)} \PY{o+ow}{or} \PY{n}{zHit} \PY{o}{\PYZgt{}} \PY{n}{zTab}\PY{p}{)}\PY{p}{)}\PY{p}{:}
    \PY{n+nb}{print}\PY{p}{(}\PY{l+s+s2}{\PYZdq{}}\PY{l+s+s2}{H0 ditolak: Residual Sugar tidak sama dengan 2.5}\PY{l+s+s2}{\PYZdq{}}\PY{p}{)}
\PY{k}{else}\PY{p}{:}
    \PY{n+nb}{print}\PY{p}{(}\PY{l+s+s2}{\PYZdq{}}\PY{l+s+s2}{H0 diterima: Residual Sugar sama dengan 2.5}\PY{l+s+s2}{\PYZdq{}}\PY{p}{)}
\end{Verbatim}
\end{tcolorbox}

    Nilai p = 0.032

    
    Nilai z = 2.148

    
    Nilai alpha = 0.05

    
    Nilai z tabel = 1.96

    
    \begin{Verbatim}[commandchars=\\\{\}]
H0 ditolak: Residual Sugar tidak sama dengan 2.5
    \end{Verbatim}

    \hypertarget{c.-nilai-rata-rata-150-baris-pertama-kolom-sulphates-bukan-0.65}{%
\subsection{c.~Nilai rata-rata 150 baris pertama kolom sulphates bukan
0.65?}\label{c.-nilai-rata-rata-150-baris-pertama-kolom-sulphates-bukan-0.65}}

    \textbf{Tentukan Hipotesis null}

\(H_0 : \mu = 0.65\)

\textbf{Tentukan Hipotesis alternatif}

\(H_1 : \mu \neq 0.65\)

\textbf{Tentukan tingkat signifikan}

\(\alpha = 0.05\)

\textbf{Menggunakan two-tailed test dan perhitungan z}

Nilai z didapatkan dari
\[ z_{Hit} = \frac{\bar{x}-\mu}{\sigma}  \sqrt{N}\]

daerah kritis yang diambil \(z < -1.96\) atau \(z > 1.96\)

\textbf{Perhitungan \(p\)-value}

\(p\)-value didapatkan dari \(2P(Z>z_{Hit})\)

\textbf{Pengambilan Keputusan}

\(H_0\) ditolak jika \(p < \alpha\) dan untuk nilai \(z\), \(z < -1.96\)
atau \(z > 1.96\)

\(H_0\) diterima jika \(p \geq \alpha\) dan untuk nilai \(z\),
\(-1.96 \leq z \leq 1.96\)

    \begin{tcolorbox}[breakable, size=fbox, boxrule=1pt, pad at break*=1mm,colback=cellbackground, colframe=cellborder]
\prompt{In}{incolor}{42}{\boxspacing}
\begin{Verbatim}[commandchars=\\\{\}]
\PY{c+c1}{\PYZsh{} Ambil 150 data pertama}
\PY{n}{testSulphate} \PY{o}{=} \PY{n}{df}\PY{o}{.}\PY{n}{head}\PY{p}{(}\PY{l+m+mi}{150}\PY{p}{)}\PY{o}{.}\PY{n}{copy}\PY{p}{(}\PY{p}{)}

\PY{c+c1}{\PYZsh{} Nilai Kepercayaan = 0.05}
\PY{n}{sig} \PY{o}{=} \PY{l+m+mf}{0.05}

\PY{c+c1}{\PYZsh{} Hitung nilai ZHit}
\PY{n}{zHit} \PY{o}{=} \PY{p}{(}\PY{n}{testSulphate}\PY{p}{[}\PY{l+s+s1}{\PYZsq{}}\PY{l+s+s1}{sulphates}\PY{l+s+s1}{\PYZsq{}}\PY{p}{]}\PY{o}{.}\PY{n}{mean}\PY{p}{(}\PY{p}{)} \PY{o}{\PYZhy{}} \PY{l+m+mf}{0.65}\PY{p}{)}\PY{o}{/}\PY{p}{(}\PY{n}{testSulphate}\PY{p}{[}\PY{l+s+s1}{\PYZsq{}}\PY{l+s+s1}{sulphates}\PY{l+s+s1}{\PYZsq{}}\PY{p}{]}\PY{o}{.}\PY{n}{std}\PY{p}{(}\PY{p}{)}\PY{o}{/}\PY{n}{np}\PY{o}{.}\PY{n}{sqrt}\PY{p}{(}\PY{l+m+mi}{150}\PY{p}{)}\PY{p}{)}

\PY{c+c1}{\PYZsh{} Hitung nilai Z Tabel}
\PY{n}{zTab} \PY{o}{=} \PY{n}{norm}\PY{o}{.}\PY{n}{ppf}\PY{p}{(}\PY{l+m+mi}{1}\PY{o}{\PYZhy{}}\PY{n}{sig}\PY{o}{/}\PY{l+m+mi}{2}\PY{p}{)}

\PY{c+c1}{\PYZsh{} Cari nilai p}
\PY{n}{PVal} \PY{o}{=} \PY{n}{norm}\PY{o}{.}\PY{n}{sf}\PY{p}{(}\PY{n+nb}{abs}\PY{p}{(}\PY{n}{zHit}\PY{p}{)}\PY{p}{)}\PY{o}{*}\PY{l+m+mi}{2}

\PY{n}{display}\PY{p}{(}\PY{n}{Markdown}\PY{p}{(}\PY{l+s+sa}{f}\PY{l+s+s2}{\PYZdq{}}\PY{l+s+s2}{Nilai p = }\PY{l+s+si}{\PYZob{}}\PY{n+nb}{round}\PY{p}{(}\PY{n}{PVal}\PY{p}{,}\PY{l+m+mi}{3}\PY{p}{)}\PY{l+s+si}{\PYZcb{}}\PY{l+s+s2}{\PYZdq{}}\PY{p}{)}\PY{p}{)}
\PY{n}{display}\PY{p}{(}\PY{n}{Markdown}\PY{p}{(}\PY{l+s+sa}{f}\PY{l+s+s2}{\PYZdq{}}\PY{l+s+s2}{Nilai z = }\PY{l+s+si}{\PYZob{}}\PY{n+nb}{round}\PY{p}{(}\PY{n}{zHit}\PY{p}{,}\PY{l+m+mi}{3}\PY{p}{)}\PY{l+s+si}{\PYZcb{}}\PY{l+s+s2}{\PYZdq{}}\PY{p}{)}\PY{p}{)}
\PY{n}{display}\PY{p}{(}\PY{n}{Markdown}\PY{p}{(}\PY{l+s+sa}{f}\PY{l+s+s2}{\PYZdq{}}\PY{l+s+s2}{Nilai alpha = }\PY{l+s+si}{\PYZob{}}\PY{n+nb}{round}\PY{p}{(}\PY{n}{sig}\PY{p}{,}\PY{l+m+mi}{3}\PY{p}{)}\PY{l+s+si}{\PYZcb{}}\PY{l+s+s2}{\PYZdq{}}\PY{p}{)}\PY{p}{)}
\PY{n}{display}\PY{p}{(}\PY{n}{Markdown}\PY{p}{(}\PY{l+s+sa}{f}\PY{l+s+s2}{\PYZdq{}}\PY{l+s+s2}{Nilai z tabel = }\PY{l+s+si}{\PYZob{}}\PY{n+nb}{round}\PY{p}{(}\PY{n}{zTab}\PY{p}{,}\PY{l+m+mi}{3}\PY{p}{)}\PY{l+s+si}{\PYZcb{}}\PY{l+s+s2}{\PYZdq{}}\PY{p}{)}\PY{p}{)}
\PY{n}{display}\PY{p}{(}\PY{n}{Markdown}\PY{p}{(}\PY{l+s+s2}{\PYZdq{}}\PY{l+s+s2}{Kesimpulan}\PY{l+s+s2}{\PYZdq{}}\PY{p}{)}\PY{p}{)}

\PY{c+c1}{\PYZsh{} Pengambilan Keputusan}
\PY{k}{if} \PY{p}{(}\PY{n}{PVal}\PY{o}{\PYZlt{}}\PY{n}{sig} \PY{o+ow}{and} \PY{p}{(}\PY{n}{zHit} \PY{o}{\PYZlt{}} \PY{o}{\PYZhy{}}\PY{n}{zTab} \PY{o+ow}{or} \PY{n}{zHit} \PY{o}{\PYZgt{}} \PY{n}{zTab}\PY{p}{)}\PY{p}{)}\PY{p}{:}
    \PY{n+nb}{print}\PY{p}{(}\PY{l+s+s2}{\PYZdq{}}\PY{l+s+s2}{H0 ditolak: Rata\PYZhy{}Rata sulphates tidak sama dengan 0.65}\PY{l+s+s2}{\PYZdq{}}\PY{p}{)}
\PY{k}{else}\PY{p}{:}
    \PY{n+nb}{print}\PY{p}{(}\PY{l+s+s2}{\PYZdq{}}\PY{l+s+s2}{H0 diterima: Rata\PYZhy{}Rata sulphates sama dengan 0.65}\PY{l+s+s2}{\PYZdq{}}\PY{p}{)}
\end{Verbatim}
\end{tcolorbox}

    \begin{Verbatim}[commandchars=\\\{\}]
0.6058666666666667
    \end{Verbatim}

    Nilai p = 0.0

    
    Nilai z = -4.965

    
    Nilai alpha = 0.05

    
    Nilai z tabel = 1.96

    
    Kesimpulan

    
    \begin{Verbatim}[commandchars=\\\{\}]
H0 ditolak: Rata-Rata sulphates tidak sama dengan 0.65
    \end{Verbatim}

    \hypertarget{d.-nilai-rata-rata-total-sulfur-dioxide-di-bawah-35}{%
\subsection{d.~Nilai rata-rata total sulfur dioxide di bawah
35?}\label{d.-nilai-rata-rata-total-sulfur-dioxide-di-bawah-35}}

    \textbf{Tentukan Hipotesis null}

\(H_0 : \mu = 35\)

\textbf{Tentukan Hipotesis alternatif}

\(H_1 : \mu < 35\)

\textbf{Tentukan tingkat signifikan}

\(\alpha = 0.05\)

\textbf{Menggunakan one-tailed test yang didekati dari kiri dan
perhitungan z}

Nilai z didapatkan dari
\[ z_{Hit} = \frac{\bar{x}-\mu}{\sigma}  \sqrt{N}\]

daerah kritis yang diambil \(z < -1.645\)

\textbf{Perhitungan \(p\)-value}

\(p\)-value didapatkan dari \(P(Z<z_{Hit})\)

\textbf{Pengambilan Keputusan}

\(H_0\) ditolak jika \(p < \alpha\) dan untuk nilai \(z\),
\(z < -1.645\)

\(H_0\) diterima jika \(p \geq \alpha\) dan untuk nilai \(z\),
\(z \geq -1.645\)

    \begin{tcolorbox}[breakable, size=fbox, boxrule=1pt, pad at break*=1mm,colback=cellbackground, colframe=cellborder]
\prompt{In}{incolor}{43}{\boxspacing}
\begin{Verbatim}[commandchars=\\\{\}]
\PY{c+c1}{\PYZsh{} Nilai Kepercayaan = 0.05}
\PY{n}{sig} \PY{o}{=} \PY{l+m+mf}{0.05}

\PY{c+c1}{\PYZsh{} Hitung nilai ZHitung}
\PY{n}{zHit} \PY{o}{=} \PY{p}{(}\PY{n}{df}\PY{p}{[}\PY{l+s+s1}{\PYZsq{}}\PY{l+s+s1}{total sulfur dioxide}\PY{l+s+s1}{\PYZsq{}}\PY{p}{]}\PY{o}{.}\PY{n}{mean}\PY{p}{(}\PY{p}{)} \PY{o}{\PYZhy{}} \PY{l+m+mi}{35}\PY{p}{)}\PY{o}{/}\PY{p}{(}\PY{n}{df}\PY{p}{[}\PY{l+s+s1}{\PYZsq{}}\PY{l+s+s1}{total sulfur dioxide}\PY{l+s+s1}{\PYZsq{}}\PY{p}{]}\PY{o}{.}\PY{n}{std}\PY{p}{(}\PY{p}{)}\PY{o}{/}\PY{n}{np}\PY{o}{.}\PY{n}{sqrt}\PY{p}{(}\PY{l+m+mi}{1000}\PY{p}{)}\PY{p}{)}

\PY{c+c1}{\PYZsh{} Hitung nilai Z tabel}
\PY{n}{zTab} \PY{o}{=} \PY{n}{norm}\PY{o}{.}\PY{n}{ppf}\PY{p}{(}\PY{n}{sig}\PY{p}{)}

\PY{c+c1}{\PYZsh{} Hitung nilai p}
\PY{n}{PVal} \PY{o}{=} \PY{n}{norm}\PY{o}{.}\PY{n}{sf}\PY{p}{(}\PY{n+nb}{abs}\PY{p}{(}\PY{n}{zHit}\PY{p}{)}\PY{p}{)}

\PY{n}{display}\PY{p}{(}\PY{n}{Markdown}\PY{p}{(}\PY{l+s+sa}{f}\PY{l+s+s2}{\PYZdq{}}\PY{l+s+s2}{Nilai p = }\PY{l+s+si}{\PYZob{}}\PY{n+nb}{round}\PY{p}{(}\PY{n}{PVal}\PY{p}{,}\PY{l+m+mi}{3}\PY{p}{)}\PY{l+s+si}{\PYZcb{}}\PY{l+s+s2}{\PYZdq{}}\PY{p}{)}\PY{p}{)}
\PY{n}{display}\PY{p}{(}\PY{n}{Markdown}\PY{p}{(}\PY{l+s+sa}{f}\PY{l+s+s2}{\PYZdq{}}\PY{l+s+s2}{Nilai z = }\PY{l+s+si}{\PYZob{}}\PY{n+nb}{round}\PY{p}{(}\PY{n}{zHit}\PY{p}{,}\PY{l+m+mi}{3}\PY{p}{)}\PY{l+s+si}{\PYZcb{}}\PY{l+s+s2}{\PYZdq{}}\PY{p}{)}\PY{p}{)}
\PY{n}{display}\PY{p}{(}\PY{n}{Markdown}\PY{p}{(}\PY{l+s+sa}{f}\PY{l+s+s2}{\PYZdq{}}\PY{l+s+s2}{Nilai alpha = }\PY{l+s+si}{\PYZob{}}\PY{n+nb}{round}\PY{p}{(}\PY{n}{sig}\PY{p}{,}\PY{l+m+mi}{3}\PY{p}{)}\PY{l+s+si}{\PYZcb{}}\PY{l+s+s2}{\PYZdq{}}\PY{p}{)}\PY{p}{)}
\PY{n}{display}\PY{p}{(}\PY{n}{Markdown}\PY{p}{(}\PY{l+s+sa}{f}\PY{l+s+s2}{\PYZdq{}}\PY{l+s+s2}{Nilai z tabel = }\PY{l+s+si}{\PYZob{}}\PY{n+nb}{round}\PY{p}{(}\PY{n}{zTab}\PY{p}{,}\PY{l+m+mi}{3}\PY{p}{)}\PY{l+s+si}{\PYZcb{}}\PY{l+s+s2}{\PYZdq{}}\PY{p}{)}\PY{p}{)}
\PY{n}{display}\PY{p}{(}\PY{n}{Markdown}\PY{p}{(}\PY{l+s+s2}{\PYZdq{}}\PY{l+s+s2}{Kesimpulan}\PY{l+s+s2}{\PYZdq{}}\PY{p}{)}\PY{p}{)}

\PY{c+c1}{\PYZsh{} Pengambilan Keputusan}
\PY{k}{if} \PY{p}{(}\PY{n}{PVal} \PY{o}{\PYZlt{}} \PY{n}{sig} \PY{o+ow}{and} \PY{n}{zHit} \PY{o}{\PYZlt{}} \PY{n}{zTab}\PY{p}{)}\PY{p}{:}
    \PY{n+nb}{print}\PY{p}{(}\PY{l+s+s2}{\PYZdq{}}\PY{l+s+s2}{H0 ditolak: rata\PYZhy{}rata total sulfur dioxide di bawah 35}\PY{l+s+s2}{\PYZdq{}}\PY{p}{)}
\PY{k}{else}\PY{p}{:}
    \PY{n+nb}{print}\PY{p}{(}\PY{l+s+s2}{\PYZdq{}}\PY{l+s+s2}{H0 diterima: rata\PYZhy{}rata total sulfur dioxide di atas 35}\PY{l+s+s2}{\PYZdq{}}\PY{p}{)}
\end{Verbatim}
\end{tcolorbox}

    Nilai p = 0.0

    
    Nilai z = 16.786

    
    Nilai alpha = 0.05

    
    Nilai z tabel = -1.645

    
    Kesimpulan

    
    \begin{Verbatim}[commandchars=\\\{\}]
H0 diterima: rata-rata total sulfur dioxide di atas 35
    \end{Verbatim}

    \hypertarget{e.-proporsi-nilai-total-sulfat-dioxide-yang-lebih-dari-40-adalah-tidak-sama-dengan-50}\label{e.-proporsi-nilai-total-sulfat-dioxide-yang-lebih-dari-40-adalah-tidak-sama-dengan-50}}

    \textbf{Tentukan Hipotesis null}

\(H_0 : p = 0.5\)

\textbf{Tentukan Hipotesis alternatif}

\(H_1 : p \neq 0.05\)

\textbf{Tentukan tingkat signifikan}

\(\alpha = 0.05\)

\textbf{Menggunakan two-tailed test dan perhitungan z}

Nilai z didapatkan dari
\[ z_{Hit} = \frac{\hat{p}-p_0}{\sqrt{\frac{p_0q_0}{n}}}\]

daerah kritis yang diambil \(z_{Hit} < -z_{Tab}\) or
\(z_{Hit}> z_{Tab}\)

    \begin{tcolorbox}[breakable, size=fbox, boxrule=1pt, pad at break*=1mm,colback=cellbackground, colframe=cellborder]
\prompt{In}{incolor}{44}{\boxspacing}
\begin{Verbatim}[commandchars=\\\{\}]
\PY{c+c1}{\PYZsh{} Pecah data untuk nilai total sulfur dioxide yang lebih dari 40}
\PY{n}{xMoreThan40} \PY{o}{=} \PY{n}{df}\PY{o}{.}\PY{n}{loc}\PY{p}{[}\PY{n}{df}\PY{p}{[}\PY{l+s+s1}{\PYZsq{}}\PY{l+s+s1}{total sulfur dioxide}\PY{l+s+s1}{\PYZsq{}}\PY{p}{]} \PY{o}{\PYZgt{}} \PY{l+m+mi}{40}\PY{p}{]}
\PY{n}{pTopi} \PY{o}{=} \PY{n+nb}{len}\PY{p}{(}\PY{n}{xMoreThan40}\PY{p}{)}\PY{o}{/}\PY{l+m+mi}{1000}

\PY{n}{sig} \PY{o}{=} \PY{l+m+mf}{0.05}
\PY{n}{zHit} \PY{o}{=} \PY{p}{(}\PY{n}{pTopi}\PY{o}{\PYZhy{}}\PY{l+m+mf}{0.5}\PY{p}{)}\PY{o}{/}\PY{p}{(}\PY{n}{np}\PY{o}{.}\PY{n}{sqrt}\PY{p}{(}\PY{l+m+mf}{0.5}\PY{o}{*}\PY{o}{*}\PY{l+m+mi}{2}\PY{o}{/}\PY{l+m+mi}{1000}\PY{p}{)}\PY{p}{)}
\PY{n}{zTab} \PY{o}{=} \PY{n}{norm}\PY{o}{.}\PY{n}{ppf}\PY{p}{(}\PY{l+m+mi}{1}\PY{o}{\PYZhy{}}\PY{n}{sig}\PY{o}{/}\PY{l+m+mi}{2}\PY{p}{)}
\PY{n}{PVal} \PY{o}{=} \PY{n}{scipy}\PY{o}{.}\PY{n}{stats}\PY{o}{.}\PY{n}{norm}\PY{o}{.}\PY{n}{sf}\PY{p}{(}\PY{n+nb}{abs}\PY{p}{(}\PY{n}{zHit}\PY{p}{)}\PY{p}{)} \PY{o}{*} \PY{l+m+mi}{2}

\PY{n}{display}\PY{p}{(}\PY{n}{Markdown}\PY{p}{(}\PY{l+s+sa}{f}\PY{l+s+s2}{\PYZdq{}}\PY{l+s+s2}{Nilai p = }\PY{l+s+si}{\PYZob{}}\PY{n+nb}{round}\PY{p}{(}\PY{n}{PVal}\PY{p}{,}\PY{l+m+mi}{3}\PY{p}{)}\PY{l+s+si}{\PYZcb{}}\PY{l+s+s2}{\PYZdq{}}\PY{p}{)}\PY{p}{)}
\PY{n}{display}\PY{p}{(}\PY{n}{Markdown}\PY{p}{(}\PY{l+s+sa}{f}\PY{l+s+s2}{\PYZdq{}}\PY{l+s+s2}{Nilai z = }\PY{l+s+si}{\PYZob{}}\PY{n+nb}{round}\PY{p}{(}\PY{n}{zHit}\PY{p}{,}\PY{l+m+mi}{3}\PY{p}{)}\PY{l+s+si}{\PYZcb{}}\PY{l+s+s2}{\PYZdq{}}\PY{p}{)}\PY{p}{)}
\PY{n}{display}\PY{p}{(}\PY{n}{Markdown}\PY{p}{(}\PY{l+s+sa}{f}\PY{l+s+s2}{\PYZdq{}}\PY{l+s+s2}{Nilai alpha = }\PY{l+s+si}{\PYZob{}}\PY{n+nb}{round}\PY{p}{(}\PY{n}{sig}\PY{p}{,}\PY{l+m+mi}{3}\PY{p}{)}\PY{l+s+si}{\PYZcb{}}\PY{l+s+s2}{\PYZdq{}}\PY{p}{)}\PY{p}{)}
\PY{n}{display}\PY{p}{(}\PY{n}{Markdown}\PY{p}{(}\PY{l+s+sa}{f}\PY{l+s+s2}{\PYZdq{}}\PY{l+s+s2}{Nilai z tabel = }\PY{l+s+si}{\PYZob{}}\PY{n+nb}{round}\PY{p}{(}\PY{n}{zTab}\PY{p}{,}\PY{l+m+mi}{3}\PY{p}{)}\PY{l+s+si}{\PYZcb{}}\PY{l+s+s2}{\PYZdq{}}\PY{p}{)}\PY{p}{)}
\PY{n}{display}\PY{p}{(}\PY{n}{Markdown}\PY{p}{(}\PY{l+s+s2}{\PYZdq{}}\PY{l+s+s2}{Kesimpulan}\PY{l+s+s2}{\PYZdq{}}\PY{p}{)}\PY{p}{)}

\PY{k}{if} \PY{p}{(}\PY{n}{PVal} \PY{o}{\PYZlt{}} \PY{n}{sig} \PY{o+ow}{and} \PY{p}{(}\PY{n}{zHit} \PY{o}{\PYZlt{}} \PY{o}{\PYZhy{}}\PY{n}{zTab} \PY{o+ow}{or} \PY{n}{zHit}\PY{o}{\PYZgt{}} \PY{n}{zTab}\PY{p}{)}\PY{p}{)}\PY{p}{:}
    \PY{n+nb}{print}\PY{p}{(}\PY{l+s+s2}{\PYZdq{}}\PY{l+s+s2}{H0 ditolak : p != 0.5}\PY{l+s+s2}{\PYZdq{}}\PY{p}{)}
\PY{k}{else}\PY{p}{:}
    \PY{n+nb}{print}\PY{p}{(}\PY{l+s+s2}{\PYZdq{}}\PY{l+s+s2}{H0 diterima : p = 0.5}\PY{l+s+s2}{\PYZdq{}}\PY{p}{)}
\end{Verbatim}
\end{tcolorbox}

    Nilai p = 0.448

    
    Nilai z = 0.759

    
    Nilai alpha = 0.05

    
    Nilai z tabel = 1.96

    
    Kesimpulan

    
    \begin{Verbatim}[commandchars=\\\{\}]
H0 diterima : p = 0.5
    \end{Verbatim}

    \hypertarget{soal-5}{%
\section{SOAL 5}\label{soal-5}}

    \hypertarget{a.-data-kolom-fixed-acidity-dibagi-2-sama-rata-bagian-awal-dan-bagian-akhir-kolom.-benarkah-rata-rata-kedua-bagian-tersebut-sama}{%
\subsection{a. Data kolom fixed acidity dibagi 2 sama rata: bagian awal
dan bagian akhir kolom. Benarkah rata-rata kedua bagian tersebut
sama?}\label{a.-data-kolom-fixed-acidity-dibagi-2-sama-rata-bagian-awal-dan-bagian-akhir-kolom.-benarkah-rata-rata-kedua-bagian-tersebut-sama}}

    \textbf{Tentukan Hipotesis null}

\(H_0 : \mu_1-\mu_2 = 0\)

\textbf{Tentukan Hipotesis alternatif}

\(H_1 : \mu_1-\mu_2 \neq 0\)

\textbf{Tentukan tingkat signifikan}

\(\alpha = 0.05\)

\textbf{Menggunakan two-tailed test dan perhitungan z}

Nilai z didapatkan dari
\[ z_{Hit} = \frac{\bar{x}_1-\bar{x}_2 - \bar{d}}{\sqrt{\frac{\sigma_1^2}{n_1} + \frac{\sigma_2^2}{n_2}}}\]

Daerah Kritis = \(z < -1.96 \text{ atau } z > 1.96\)

    \begin{tcolorbox}[breakable, size=fbox, boxrule=1pt, pad at break*=1mm,colback=cellbackground, colframe=cellborder]
\prompt{In}{incolor}{45}{\boxspacing}
\begin{Verbatim}[commandchars=\\\{\}]
\PY{n}{fixedAcidityAwal} \PY{o}{=} \PY{n}{df}\PY{p}{[}\PY{l+m+mi}{0}\PY{p}{:}\PY{l+m+mi}{500}\PY{p}{]}\PY{o}{.}\PY{n}{copy}\PY{p}{(}\PY{p}{)}
\PY{n}{fixedAcidityAkhir} \PY{o}{=} \PY{n}{df}\PY{p}{[}\PY{l+m+mi}{500}\PY{p}{:}\PY{l+m+mi}{1000}\PY{p}{]}\PY{o}{.}\PY{n}{copy}\PY{p}{(}\PY{p}{)}
\PY{n}{m1} \PY{o}{=} \PY{n}{fixedAcidityAwal}\PY{p}{[}\PY{l+s+s1}{\PYZsq{}}\PY{l+s+s1}{fixed acidity}\PY{l+s+s1}{\PYZsq{}}\PY{p}{]}\PY{o}{.}\PY{n}{mean}\PY{p}{(}\PY{p}{)}
\PY{n}{m2} \PY{o}{=} \PY{n}{fixedAcidityAkhir}\PY{p}{[}\PY{l+s+s1}{\PYZsq{}}\PY{l+s+s1}{fixed acidity}\PY{l+s+s1}{\PYZsq{}}\PY{p}{]}\PY{o}{.}\PY{n}{mean}\PY{p}{(}\PY{p}{)}

\PY{c+c1}{\PYZsh{} H0: m1 \PYZhy{} m2 = 0}
\PY{c+c1}{\PYZsh{} H1: m1 \PYZhy{} m2 != 0}
\PY{c+c1}{\PYZsh{} Derajat Kepercayaan = 0.05}
\PY{c+c1}{\PYZsh{} Daerah Kritis = z \PYZlt{} \PYZhy{}1.96 atau z \PYZgt{} 1.96}

\PY{c+c1}{\PYZsh{} Computation}
\PY{n}{sig} \PY{o}{=} \PY{l+m+mf}{0.05}
\PY{n}{z} \PY{o}{=} \PY{p}{(}\PY{n}{m1}\PY{o}{\PYZhy{}}\PY{n}{m2}\PY{o}{\PYZhy{}}\PY{l+m+mi}{0}\PY{p}{)}\PY{o}{/}\PY{p}{(}\PY{n}{np}\PY{o}{.}\PY{n}{sqrt}\PY{p}{(}\PY{p}{(}\PY{n}{fixedAcidityAwal}\PY{p}{[}\PY{l+s+s1}{\PYZsq{}}\PY{l+s+s1}{fixed acidity}\PY{l+s+s1}{\PYZsq{}}\PY{p}{]}\PY{o}{.}\PY{n}{std}\PY{p}{(}\PY{p}{)}\PY{o}{*}\PY{o}{*}\PY{l+m+mi}{2}\PY{o}{/}\PY{l+m+mi}{500}\PY{p}{)} \PY{o}{+} \PY{p}{(}\PY{n}{fixedAcidityAkhir}\PY{p}{[}\PY{l+s+s1}{\PYZsq{}}\PY{l+s+s1}{fixed acidity}\PY{l+s+s1}{\PYZsq{}}\PY{p}{]}\PY{o}{.}\PY{n}{std}\PY{p}{(}\PY{p}{)}\PY{o}{*}\PY{o}{*}\PY{l+m+mi}{2}\PY{o}{/}\PY{l+m+mi}{500}\PY{p}{)}\PY{p}{)}\PY{p}{)}
\PY{n}{zTab} \PY{o}{=} \PY{n}{norm}\PY{o}{.}\PY{n}{ppf}\PY{p}{(}\PY{l+m+mi}{1}\PY{o}{\PYZhy{}}\PY{n}{sig}\PY{o}{/}\PY{l+m+mi}{2}\PY{p}{)}
\PY{n}{PVal} \PY{o}{=} \PY{n}{scipy}\PY{o}{.}\PY{n}{stats}\PY{o}{.}\PY{n}{norm}\PY{o}{.}\PY{n}{sf}\PY{p}{(}\PY{n+nb}{abs}\PY{p}{(}\PY{n}{z}\PY{p}{)}\PY{p}{)}\PY{o}{*}\PY{l+m+mi}{2}

\PY{n}{display}\PY{p}{(}\PY{n}{Markdown}\PY{p}{(}\PY{l+s+sa}{f}\PY{l+s+s2}{\PYZdq{}}\PY{l+s+s2}{Nilai p = }\PY{l+s+si}{\PYZob{}}\PY{n+nb}{round}\PY{p}{(}\PY{n}{PVal}\PY{p}{,}\PY{l+m+mi}{3}\PY{p}{)}\PY{l+s+si}{\PYZcb{}}\PY{l+s+s2}{\PYZdq{}}\PY{p}{)}\PY{p}{)}
\PY{n}{display}\PY{p}{(}\PY{n}{Markdown}\PY{p}{(}\PY{l+s+sa}{f}\PY{l+s+s2}{\PYZdq{}}\PY{l+s+s2}{Nilai z = }\PY{l+s+si}{\PYZob{}}\PY{n+nb}{round}\PY{p}{(}\PY{n}{z}\PY{p}{,}\PY{l+m+mi}{3}\PY{p}{)}\PY{l+s+si}{\PYZcb{}}\PY{l+s+s2}{\PYZdq{}}\PY{p}{)}\PY{p}{)}
\PY{n}{display}\PY{p}{(}\PY{n}{Markdown}\PY{p}{(}\PY{l+s+sa}{f}\PY{l+s+s2}{\PYZdq{}}\PY{l+s+s2}{Nilai alpha = }\PY{l+s+si}{\PYZob{}}\PY{n+nb}{round}\PY{p}{(}\PY{n}{sig}\PY{p}{,}\PY{l+m+mi}{3}\PY{p}{)}\PY{l+s+si}{\PYZcb{}}\PY{l+s+s2}{\PYZdq{}}\PY{p}{)}\PY{p}{)}
\PY{n}{display}\PY{p}{(}\PY{n}{Markdown}\PY{p}{(}\PY{l+s+sa}{f}\PY{l+s+s2}{\PYZdq{}}\PY{l+s+s2}{Nilai z tabel = }\PY{l+s+si}{\PYZob{}}\PY{n+nb}{round}\PY{p}{(}\PY{n}{zTab}\PY{p}{,}\PY{l+m+mi}{3}\PY{p}{)}\PY{l+s+si}{\PYZcb{}}\PY{l+s+s2}{\PYZdq{}}\PY{p}{)}\PY{p}{)}
\PY{n}{display}\PY{p}{(}\PY{n}{Markdown}\PY{p}{(}\PY{l+s+s2}{\PYZdq{}}\PY{l+s+s2}{Kesimpulan}\PY{l+s+s2}{\PYZdq{}}\PY{p}{)}\PY{p}{)}

\PY{k}{if} \PY{p}{(}\PY{n}{PVal} \PY{o}{\PYZlt{}} \PY{n}{sig} \PY{o+ow}{and} \PY{p}{(}\PY{n}{z} \PY{o}{\PYZlt{}} \PY{o}{\PYZhy{}}\PY{n}{zTab} \PY{o+ow}{or} \PY{n}{z} \PY{o}{\PYZgt{}} \PY{n}{zTab}\PY{p}{)}\PY{p}{)}\PY{p}{:}
    \PY{n+nb}{print}\PY{p}{(}\PY{l+s+s2}{\PYZdq{}}\PY{l+s+s2}{H0 ditolak, bagian awal tidak sama dengan bagian akhir rata2nya}\PY{l+s+s2}{\PYZdq{}}\PY{p}{)}
\PY{k}{else}\PY{p}{:}
    \PY{n+nb}{print}\PY{p}{(}\PY{l+s+s2}{\PYZdq{}}\PY{l+s+s2}{H0 diterima, bagian awal sama dengan bagian akhir rata2nya}\PY{l+s+s2}{\PYZdq{}}\PY{p}{)}
\end{Verbatim}
\end{tcolorbox}

    Nilai p = 0.979

    
    Nilai z = 0.026

    
    Nilai alpha = 0.05

    
    Nilai z tabel = 1.96

    
    Kesimpulan

    
    \begin{Verbatim}[commandchars=\\\{\}]
H0 diterima, bagian awal sama dengan bagian akhir rata2nya
    \end{Verbatim}

    \hypertarget{b.-data-kolom-chlorides-dibagi-2-sama-rata-bagian-awal-dan-bagian-akhir-kolom.-benarkah-rata-rata-bagian-awal-lebih-besar-daripada-bagian-akhir-sebesar-0.001}{%
\subsection{b. Data kolom chlorides dibagi 2 sama rata: bagian awal dan
bagian akhir kolom. Benarkah rata-rata bagian awal lebih besar daripada
bagian akhir sebesar
0.001?}\label{b.-data-kolom-chlorides-dibagi-2-sama-rata-bagian-awal-dan-bagian-akhir-kolom.-benarkah-rata-rata-bagian-awal-lebih-besar-daripada-bagian-akhir-sebesar-0.001}}

    \textbf{Tentukan Hipotesis null}

\(H_0 : \mu_1 - \mu_2 = 0.001\)

\textbf{Tentukan Hipotesis alternatif}

\(H_1 : \mu_1 - \mu_2 > 0.001\)

\textbf{Tentukan tingkat signifikan}

\(\alpha = 0.05\)

\textbf{Menggunakan one-tailed test dan perhitungan z}

Nilai z didapatkan dari
\[ z_{Hit} = \frac{\bar{x}_1-\bar{x}_2 - \bar{d}}{\sqrt{\frac{\sigma_1^2}{n_1} + \frac{\sigma_2^2}{n_2}}}\]

Ambil daerah kritis \(z > 1.645\)

    \begin{tcolorbox}[breakable, size=fbox, boxrule=1pt, pad at break*=1mm,colback=cellbackground, colframe=cellborder]
\prompt{In}{incolor}{46}{\boxspacing}
\begin{Verbatim}[commandchars=\\\{\}]
\PY{c+c1}{\PYZsh{} ini emang ada 2 ya critical valuenya?}
\PY{n}{chloridesAwal} \PY{o}{=} \PY{n}{df}\PY{p}{[}\PY{l+m+mi}{0}\PY{p}{:}\PY{l+m+mi}{500}\PY{p}{]}\PY{o}{.}\PY{n}{copy}\PY{p}{(}\PY{p}{)}
\PY{n}{chloridesAkhir} \PY{o}{=} \PY{n}{df}\PY{p}{[}\PY{l+m+mi}{500}\PY{p}{:}\PY{l+m+mi}{1000}\PY{p}{]}\PY{o}{.}\PY{n}{copy}\PY{p}{(}\PY{p}{)}
\PY{n}{m1} \PY{o}{=} \PY{n}{chloridesAwal}\PY{p}{[}\PY{l+s+s1}{\PYZsq{}}\PY{l+s+s1}{chlorides}\PY{l+s+s1}{\PYZsq{}}\PY{p}{]}\PY{o}{.}\PY{n}{mean}\PY{p}{(}\PY{p}{)}
\PY{n}{m2} \PY{o}{=} \PY{n}{chloridesAkhir}\PY{p}{[}\PY{l+s+s1}{\PYZsq{}}\PY{l+s+s1}{chlorides}\PY{l+s+s1}{\PYZsq{}}\PY{p}{]}\PY{o}{.}\PY{n}{mean}\PY{p}{(}\PY{p}{)}

\PY{c+c1}{\PYZsh{} H0: m1 \PYZhy{} m2 = 0.001}
\PY{c+c1}{\PYZsh{} H1: m1 \PYZhy{} m2 \PYZgt{} 0.001}
\PY{c+c1}{\PYZsh{} Derajat Kepercayaan: 0.05}
\PY{c+c1}{\PYZsh{} Derajat Kebebasan = 1000 \PYZhy{} 2 = 998}
\PY{c+c1}{\PYZsh{} Critical Value z \PYZgt{} 1.645}

\PY{c+c1}{\PYZsh{} Computation}
\PY{n}{sig} \PY{o}{=} \PY{l+m+mf}{0.05}
\PY{n}{zHit} \PY{o}{=} \PY{p}{(}\PY{n}{m1}\PY{o}{\PYZhy{}}\PY{n}{m2}\PY{o}{\PYZhy{}}\PY{l+m+mf}{0.001}\PY{p}{)}\PY{o}{/}\PY{p}{(}\PY{n}{np}\PY{o}{.}\PY{n}{sqrt}\PY{p}{(}\PY{p}{(}\PY{n}{chloridesAwal}\PY{p}{[}\PY{l+s+s1}{\PYZsq{}}\PY{l+s+s1}{chlorides}\PY{l+s+s1}{\PYZsq{}}\PY{p}{]}\PY{o}{.}\PY{n}{std}\PY{p}{(}\PY{p}{)}\PY{o}{*}\PY{o}{*}\PY{l+m+mi}{2}\PY{o}{/}\PY{l+m+mi}{500}\PY{p}{)} \PY{o}{+} \PY{p}{(}\PY{n}{fixedAcidityAkhir}\PY{p}{[}\PY{l+s+s1}{\PYZsq{}}\PY{l+s+s1}{chlorides}\PY{l+s+s1}{\PYZsq{}}\PY{p}{]}\PY{o}{.}\PY{n}{std}\PY{p}{(}\PY{p}{)}\PY{o}{*}\PY{o}{*}\PY{l+m+mi}{2}\PY{o}{/}\PY{l+m+mi}{500}\PY{p}{)}\PY{p}{)}\PY{p}{)}
\PY{n}{zTab} \PY{o}{=} \PY{n}{norm}\PY{o}{.}\PY{n}{ppf}\PY{p}{(}\PY{l+m+mi}{1}\PY{o}{\PYZhy{}}\PY{n}{sig}\PY{p}{)}
\PY{n}{PVal} \PY{o}{=} \PY{n}{norm}\PY{o}{.}\PY{n}{sf}\PY{p}{(}\PY{n}{zHit}\PY{p}{)} \PY{o}{*} \PY{l+m+mi}{2}

\PY{n}{display}\PY{p}{(}\PY{n}{Markdown}\PY{p}{(}\PY{l+s+sa}{f}\PY{l+s+s2}{\PYZdq{}}\PY{l+s+s2}{Nilai p = }\PY{l+s+si}{\PYZob{}}\PY{n+nb}{round}\PY{p}{(}\PY{n}{PVal}\PY{p}{,}\PY{l+m+mi}{3}\PY{p}{)}\PY{l+s+si}{\PYZcb{}}\PY{l+s+s2}{\PYZdq{}}\PY{p}{)}\PY{p}{)}
\PY{n}{display}\PY{p}{(}\PY{n}{Markdown}\PY{p}{(}\PY{l+s+sa}{f}\PY{l+s+s2}{\PYZdq{}}\PY{l+s+s2}{Nilai z = }\PY{l+s+si}{\PYZob{}}\PY{n+nb}{round}\PY{p}{(}\PY{n}{zHit}\PY{p}{,}\PY{l+m+mi}{3}\PY{p}{)}\PY{l+s+si}{\PYZcb{}}\PY{l+s+s2}{\PYZdq{}}\PY{p}{)}\PY{p}{)}
\PY{n}{display}\PY{p}{(}\PY{n}{Markdown}\PY{p}{(}\PY{l+s+sa}{f}\PY{l+s+s2}{\PYZdq{}}\PY{l+s+s2}{Nilai alpha = }\PY{l+s+si}{\PYZob{}}\PY{n+nb}{round}\PY{p}{(}\PY{n}{sig}\PY{p}{,}\PY{l+m+mi}{3}\PY{p}{)}\PY{l+s+si}{\PYZcb{}}\PY{l+s+s2}{\PYZdq{}}\PY{p}{)}\PY{p}{)}
\PY{n}{display}\PY{p}{(}\PY{n}{Markdown}\PY{p}{(}\PY{l+s+sa}{f}\PY{l+s+s2}{\PYZdq{}}\PY{l+s+s2}{Nilai z tabel = }\PY{l+s+si}{\PYZob{}}\PY{n+nb}{round}\PY{p}{(}\PY{n}{zTab}\PY{p}{,}\PY{l+m+mi}{3}\PY{p}{)}\PY{l+s+si}{\PYZcb{}}\PY{l+s+s2}{\PYZdq{}}\PY{p}{)}\PY{p}{)}
\PY{n}{display}\PY{p}{(}\PY{n}{Markdown}\PY{p}{(}\PY{l+s+s2}{\PYZdq{}}\PY{l+s+s2}{Kesimpulan}\PY{l+s+s2}{\PYZdq{}}\PY{p}{)}\PY{p}{)}

\PY{k}{if}\PY{p}{(}\PY{p}{(}\PY{n}{zHit} \PY{o}{\PYZgt{}} \PY{n}{zTab}\PY{p}{)} \PY{o+ow}{and} \PY{n}{PVal} \PY{o}{\PYZlt{}} \PY{l+m+mf}{0.05}\PY{p}{)}\PY{p}{:}
    \PY{n+nb}{print}\PY{p}{(}\PY{l+s+s2}{\PYZdq{}}\PY{l+s+s2}{H0 ditolak, bagian awal lebih besar dari bagian akhirnya }\PY{l+s+s2}{\PYZdq{}}\PY{p}{)}
\PY{k}{else}\PY{p}{:}
    \PY{n+nb}{print}\PY{p}{(}\PY{l+s+s2}{\PYZdq{}}\PY{l+s+s2}{H0 diterima, bagian awal tidak lebih besar dari bagian akhirnya}\PY{l+s+s2}{\PYZdq{}}\PY{p}{)}
\end{Verbatim}
\end{tcolorbox}

    Nilai p = 1.36

    
    Nilai z = -0.467

    
    Nilai alpha = 0.05

    
    Nilai z tabel = 1.645

    
    Kesimpulan

    
    \begin{Verbatim}[commandchars=\\\{\}]
H0 diterima, bagian awal tidak lebih besar dari bagian akhirnya
    \end{Verbatim}

    \hypertarget{c.-benarkah-rata-rata-sampel-25-baris-pertama-kolom-volatile-acidity-sama-dengan-rata-rata-25-baris-pertama-kolom-sulphates}{%
\subsection{c.~Benarkah rata-rata sampel 25 baris pertama kolom Volatile
Acidity sama dengan rata-rata 25 baris pertama kolom Sulphates
?}\label{c.-benarkah-rata-rata-sampel-25-baris-pertama-kolom-volatile-acidity-sama-dengan-rata-rata-25-baris-pertama-kolom-sulphates}}

    \textbf{Tentukan Hipotesis null}

\(H_0 : \mu_{VA} - \mu_{VS} = 0\)

\textbf{Tentukan Hipotesis alternatif}

\(H_1 : \mu_{VA} - \mu_{VS} \neq 0\)

\textbf{Tentukan tingkat signifikan}

\(\alpha = 0.05\)

\textbf{Menggunakan two tailed test dan perhitungan spooled t}

Nilai \(t\) didapatkan dari
\[ t = \frac{\bar{x}_1 - \bar{x}_2 - d_0}{s^2_p \sqrt{\frac{1}{n_1} + \frac{1}{n_2}}}\]

dan \(s^2_p\)
\[ s^2_p = \frac{s^2_1 (n_1 -1) + s^2_2 (n_2-1)}{n_1+n_2 -2} \]

dengan critical value \(t < -t_{tab}\) atau \(t > t_{tab}\)

\textbf{Perhitungan \(p\)-value}

\(p\)-value didapatkan dari \(2P(T > t)\)

    \begin{tcolorbox}[breakable, size=fbox, boxrule=1pt, pad at break*=1mm,colback=cellbackground, colframe=cellborder]
\prompt{In}{incolor}{47}{\boxspacing}
\begin{Verbatim}[commandchars=\\\{\}]
\PY{n}{splitTwentyFive} \PY{o}{=} \PY{n}{df}\PY{o}{.}\PY{n}{head}\PY{p}{(}\PY{l+m+mi}{25}\PY{p}{)}\PY{o}{.}\PY{n}{copy}\PY{p}{(}\PY{p}{)}
\PY{n}{mVA} \PY{o}{=} \PY{n}{splitTwentyFive}\PY{p}{[}\PY{l+s+s1}{\PYZsq{}}\PY{l+s+s1}{volatile acidity}\PY{l+s+s1}{\PYZsq{}}\PY{p}{]}\PY{o}{.}\PY{n}{mean}\PY{p}{(}\PY{p}{)}
\PY{n}{mS} \PY{o}{=} \PY{n}{splitTwentyFive}\PY{p}{[}\PY{l+s+s1}{\PYZsq{}}\PY{l+s+s1}{sulphates}\PY{l+s+s1}{\PYZsq{}}\PY{p}{]}\PY{o}{.}\PY{n}{mean}\PY{p}{(}\PY{p}{)}

\PY{c+c1}{\PYZsh{} H0: mVA \PYZhy{} mVS = 0}
\PY{c+c1}{\PYZsh{} H1: mVA \PYZhy{} mVS != 0}
\PY{c+c1}{\PYZsh{} Derajat Kepercayaan: 0.05}
\PY{c+c1}{\PYZsh{} Derajat Kebebasan = 25 + 25 \PYZhy{} 2 = 48}
\PY{c+c1}{\PYZsh{} Critical Value t \PYZlt{} \PYZhy{}2.011 atau t \PYZgt{} 2.011}

\PY{c+c1}{\PYZsh{} Computation}
\PY{n}{sp} \PY{o}{=} \PY{p}{(}\PY{n}{splitTwentyFive}\PY{p}{[}\PY{l+s+s1}{\PYZsq{}}\PY{l+s+s1}{volatile acidity}\PY{l+s+s1}{\PYZsq{}}\PY{p}{]}\PY{o}{.}\PY{n}{std}\PY{p}{(}\PY{p}{)}\PY{o}{*}\PY{o}{*}\PY{l+m+mi}{2} \PY{o}{*} \PY{p}{(}\PY{l+m+mi}{24}\PY{p}{)} \PY{o}{+} \PY{n}{splitTwentyFive}\PY{p}{[}\PY{l+s+s1}{\PYZsq{}}\PY{l+s+s1}{sulphates}\PY{l+s+s1}{\PYZsq{}}\PY{p}{]}\PY{o}{.}\PY{n}{std}\PY{p}{(}\PY{p}{)}\PY{o}{*}\PY{o}{*}\PY{l+m+mi}{2} \PY{o}{*} \PY{p}{(}\PY{l+m+mi}{24}\PY{p}{)}\PY{p}{)}\PY{o}{/}\PY{l+m+mi}{48}
\PY{n}{t} \PY{o}{=} \PY{p}{(}\PY{n}{mVA} \PY{o}{\PYZhy{}} \PY{n}{mS} \PY{o}{\PYZhy{}} \PY{l+m+mi}{0}\PY{p}{)}\PY{o}{/}\PY{p}{(}\PY{p}{(}\PY{n}{np}\PY{o}{.}\PY{n}{sqrt}\PY{p}{(}\PY{n}{sp}\PY{p}{)}\PY{o}{*}\PY{n}{np}\PY{o}{.}\PY{n}{sqrt}\PY{p}{(}\PY{p}{(}\PY{l+m+mi}{1}\PY{o}{/}\PY{l+m+mi}{25} \PY{o}{+} \PY{l+m+mi}{1}\PY{o}{/}\PY{l+m+mi}{25}\PY{p}{)}\PY{p}{)}\PY{p}{)}\PY{p}{)}

\PY{n}{display}\PY{p}{(}\PY{n}{Markdown}\PY{p}{(}\PY{l+s+sa}{f}\PY{l+s+s2}{\PYZdq{}}\PY{l+s+s2}{t = }\PY{l+s+si}{\PYZob{}}\PY{n+nb}{round}\PY{p}{(}\PY{n}{t}\PY{p}{,}\PY{l+m+mi}{3}\PY{p}{)}\PY{l+s+si}{\PYZcb{}}\PY{l+s+s2}{\PYZdq{}}\PY{p}{)}\PY{p}{)}
\PY{n}{ttab} \PY{o}{=} \PY{n}{scipy}\PY{o}{.}\PY{n}{stats}\PY{o}{.}\PY{n}{t}\PY{o}{.}\PY{n}{ppf}\PY{p}{(}\PY{n}{q}\PY{o}{=}\PY{l+m+mi}{1}\PY{o}{\PYZhy{}}\PY{l+m+mf}{0.05}\PY{o}{/}\PY{l+m+mi}{2}\PY{p}{,} \PY{n}{df} \PY{o}{=} \PY{l+m+mi}{48}\PY{p}{)}
\PY{n}{display}\PY{p}{(}\PY{n}{Markdown}\PY{p}{(}\PY{l+s+sa}{f}\PY{l+s+s2}{\PYZdq{}}\PY{l+s+s2}{t tabel = }\PY{l+s+si}{\PYZob{}}\PY{n+nb}{round}\PY{p}{(}\PY{n}{ttab}\PY{p}{,}\PY{l+m+mi}{3}\PY{p}{)}\PY{l+s+si}{\PYZcb{}}\PY{l+s+s2}{\PYZdq{}}\PY{p}{)}\PY{p}{)}

\PY{c+c1}{\PYZsh{}Cari nilai P}
\PY{n}{PVal} \PY{o}{=} \PY{n}{scipy}\PY{o}{.}\PY{n}{stats}\PY{o}{.}\PY{n}{t}\PY{o}{.}\PY{n}{sf}\PY{p}{(}\PY{n+nb}{abs}\PY{p}{(}\PY{n}{t}\PY{p}{)}\PY{p}{,} \PY{n}{df}\PY{o}{=}\PY{l+m+mi}{48}\PY{p}{)} \PY{o}{*} \PY{l+m+mi}{2}

\PY{n}{display}\PY{p}{(}\PY{n}{Markdown}\PY{p}{(}\PY{l+s+sa}{f}\PY{l+s+s2}{\PYZdq{}}\PY{l+s+s2}{Nilai alpha = 0.05}\PY{l+s+s2}{\PYZdq{}}\PY{p}{)}\PY{p}{)}
\PY{n}{display}\PY{p}{(}\PY{n}{Markdown}\PY{p}{(}\PY{l+s+sa}{f}\PY{l+s+s2}{\PYZdq{}}\PY{l+s+s2}{P Value = }\PY{l+s+si}{\PYZob{}}\PY{n+nb}{round}\PY{p}{(}\PY{n}{PVal}\PY{p}{,}\PY{l+m+mi}{3}\PY{p}{)}\PY{l+s+si}{\PYZcb{}}\PY{l+s+s2}{\PYZdq{}}\PY{p}{)}\PY{p}{)}

\PY{n}{display}\PY{p}{(}\PY{n}{Markdown}\PY{p}{(}\PY{l+s+s2}{\PYZdq{}}\PY{l+s+s2}{Kesimpulan:}\PY{l+s+s2}{\PYZdq{}}\PY{p}{)}\PY{p}{)}

\PY{k}{if} \PY{p}{(}\PY{n}{PVal} \PY{o}{\PYZlt{}} \PY{n}{sig} \PY{o+ow}{and} \PY{p}{(}\PY{n}{t} \PY{o}{\PYZlt{}} \PY{o}{\PYZhy{}}\PY{n}{ttab} \PY{o+ow}{or} \PY{n}{t} \PY{o}{\PYZgt{}} \PY{n}{ttab}\PY{p}{)}\PY{p}{)}\PY{p}{:}
    \PY{n+nb}{print}\PY{p}{(}\PY{l+s+s2}{\PYZdq{}}\PY{l+s+s2}{H0 ditolak, rata\PYZhy{}rata sampel 25 baris pertama kolom Volatile Acidity tidak sama dengan rata\PYZhy{}rata 25 baris pertama kolom Sulphates}\PY{l+s+s2}{\PYZdq{}}\PY{p}{)}
\PY{k}{else}\PY{p}{:}
    \PY{n+nb}{print}\PY{p}{(}\PY{l+s+s2}{\PYZdq{}}\PY{l+s+s2}{H0 diterima, rata\PYZhy{}rata sampel 25 baris pertama kolom Volatile Acidity sama dengan rata\PYZhy{}rata 25 baris pertama kolom Sulphates}\PY{l+s+s2}{\PYZdq{}}\PY{p}{)}
\end{Verbatim}
\end{tcolorbox}

    t = -2.637

    
    t tabel = 2.011

    
    Nilai alpha = 0.05

    
    P Value = 0.011

    
    Kesimpulan:

    
    \begin{Verbatim}[commandchars=\\\{\}]
H0 ditolak, rata-rata sampel 25 baris pertama kolom Volatile Acidity tidak sama
dengan rata-rata 25 baris pertama kolom Sulphates
    \end{Verbatim}

    \hypertarget{d.-bagian-awal-kolom-residual-sugar-memiliki-variansi-yang-sama-dengan-bagian-akhirnya}{%
\subsection{d.~Bagian awal kolom residual sugar memiliki variansi yang
sama dengan bagian
akhirnya?}\label{d.-bagian-awal-kolom-residual-sugar-memiliki-variansi-yang-sama-dengan-bagian-akhirnya}}

    \textbf{Tentukan Hipotesis null}

\(H_0 : v1 = v2\)

\textbf{Tentukan Hipotesis alternatif}

\(H_1 : v1 \neq v2\)

\textbf{Tentukan tingkat signifikan}

\(\alpha = 0.05\)

\textbf{Menggunakan uji two tailed f test}

Perhitungan nilai \(f\) \[f = \frac{s^2_1}{s^2_2}\]

ambil daerah kritis \(f < f_{1-\alpha/2}(v_1,v_2)\) dan
\(f > f_{\alpha / 2}(v_1,v_2)\)

degan \(v_1 = n_1-1, v_2 = n_2-1\)

    \begin{tcolorbox}[breakable, size=fbox, boxrule=1pt, pad at break*=1mm,colback=cellbackground, colframe=cellborder]
\prompt{In}{incolor}{48}{\boxspacing}
\begin{Verbatim}[commandchars=\\\{\}]
\PY{c+c1}{\PYZsh{} H0 : v1 = v2}
\PY{c+c1}{\PYZsh{} H1 : v1 != v2}

\PY{n}{rs} \PY{o}{=} \PY{n}{df}\PY{p}{[}\PY{l+s+s1}{\PYZsq{}}\PY{l+s+s1}{residual sugar}\PY{l+s+s1}{\PYZsq{}}\PY{p}{]}\PY{o}{.}\PY{n}{copy}\PY{p}{(}\PY{p}{)}
\PY{n}{r1} \PY{o}{=} \PY{n}{rs}\PY{p}{[}\PY{p}{:}\PY{l+m+mi}{500}\PY{p}{]}
\PY{n}{r2} \PY{o}{=} \PY{n}{rs}\PY{p}{[}\PY{l+m+mi}{500}\PY{p}{:}\PY{l+m+mi}{1000}\PY{p}{]}

\PY{c+c1}{\PYZsh{}cari critical region}
\PY{n}{batas1} \PY{o}{=} \PY{n}{scipy}\PY{o}{.}\PY{n}{stats}\PY{o}{.}\PY{n}{f}\PY{o}{.}\PY{n}{ppf}\PY{p}{(}\PY{n}{q}\PY{o}{=}\PY{l+m+mi}{1}\PY{o}{\PYZhy{}}\PY{l+m+mf}{0.975}\PY{p}{,} \PY{n}{dfn}\PY{o}{=}\PY{l+m+mi}{500}\PY{o}{\PYZhy{}}\PY{l+m+mi}{1}\PY{p}{,} \PY{n}{dfd}\PY{o}{=}\PY{l+m+mi}{500}\PY{o}{\PYZhy{}}\PY{l+m+mi}{1}\PY{p}{)}
\PY{c+c1}{\PYZsh{} print(batas1)}
\PY{n}{batas2} \PY{o}{=} \PY{n}{scipy}\PY{o}{.}\PY{n}{stats}\PY{o}{.}\PY{n}{f}\PY{o}{.}\PY{n}{ppf}\PY{p}{(}\PY{n}{q}\PY{o}{=}\PY{l+m+mf}{0.975}\PY{p}{,} \PY{n}{dfn}\PY{o}{=}\PY{l+m+mi}{500}\PY{o}{\PYZhy{}}\PY{l+m+mi}{1}\PY{p}{,} \PY{n}{dfd}\PY{o}{=}\PY{l+m+mi}{500}\PY{o}{\PYZhy{}}\PY{l+m+mi}{1}\PY{p}{)}
\PY{c+c1}{\PYZsh{} print(batas2)}

\PY{c+c1}{\PYZsh{}cari f nya}
\PY{n}{s1} \PY{o}{=} \PY{n}{r1}\PY{o}{.}\PY{n}{var}\PY{p}{(}\PY{p}{)}
\PY{n}{s2} \PY{o}{=} \PY{n}{r2}\PY{o}{.}\PY{n}{var}\PY{p}{(}\PY{p}{)}
\PY{n}{f} \PY{o}{=} \PY{n}{s1}\PY{o}{/}\PY{n}{s2}

\PY{n}{display}\PY{p}{(}\PY{n}{Markdown}\PY{p}{(}\PY{l+s+sa}{f}\PY{l+s+s2}{\PYZdq{}}\PY{l+s+s2}{Nilai f: }\PY{l+s+si}{\PYZob{}}\PY{n+nb}{round}\PY{p}{(}\PY{n}{f}\PY{p}{,}\PY{l+m+mi}{3}\PY{p}{)}\PY{l+s+si}{\PYZcb{}}\PY{l+s+s2}{\PYZdq{}}\PY{p}{)}\PY{p}{)}
\PY{n}{display}\PY{p}{(}\PY{n}{Markdown}\PY{p}{(}\PY{l+s+s2}{\PYZdq{}}\PY{l+s+s2}{Kesimpulan:}\PY{l+s+s2}{\PYZdq{}}\PY{p}{)}\PY{p}{)}
\PY{k}{if} \PY{p}{(}\PY{n}{f}\PY{o}{\PYZlt{}}\PY{n}{batas1} \PY{o+ow}{or} \PY{n}{f}\PY{o}{\PYZgt{}}\PY{n}{batas2}\PY{p}{)}\PY{p}{:}
    \PY{n+nb}{print}\PY{p}{(}\PY{l+s+s2}{\PYZdq{}}\PY{l+s+s2}{H0 ditolak : v1 != v2}\PY{l+s+s2}{\PYZdq{}}\PY{p}{)}
\PY{k}{else}\PY{p}{:}
    \PY{n+nb}{print}\PY{p}{(}\PY{l+s+s2}{\PYZdq{}}\PY{l+s+s2}{H0 diterima : v1 = v2}\PY{l+s+s2}{\PYZdq{}}\PY{p}{)}
\end{Verbatim}
\end{tcolorbox}

    Nilai f: 0.942

    
    Kesimpulan:

    
    \begin{Verbatim}[commandchars=\\\{\}]
H0 diterima : v1 = v2
    \end{Verbatim}

    \hypertarget{e.-proporsi-nilai-setengah-bagian-awal-alcohol-yang-lebih-dari-7-adalah-lebih-besar-daripada-proporsi-nilai-yang-sama-di-setengah-bagian-akhir-alcohol}{%
\subsection{e. Proporsi nilai setengah bagian awal alcohol yang lebih
dari 7, adalah lebih besar daripada, proporsi nilai yang sama di
setengah bagian akhir
alcohol?}\label{e.-proporsi-nilai-setengah-bagian-awal-alcohol-yang-lebih-dari-7-adalah-lebih-besar-daripada-proporsi-nilai-yang-sama-di-setengah-bagian-akhir-alcohol}}

    \textbf{Tentukan Hipotesis null}

\(H_0 : p1 = p2\)

\textbf{Tentukan Hipotesis alternatif}

\(H_1 : p1 \neq p2\)

\textbf{Tentukan tingkat signifikan}

\(\alpha = 0.05\)

\textbf{Two tailed test dengan z untuk testing p1 = p2}

Nilai \(z\)
\[ z= \frac{\hat{p}_1 - \hat{p}_2}{\sqrt{\hat{p}\hat{q}(1/n_1 + 1/n_2)}}\]

dengan \(\hat{p}_1 = \frac{x_1}{n_1}, \hat{p}_2 = \frac{x_2}{n_2}\) dan
\(\hat{p} = \frac{x_1 + x_2}{n_1 + n_2}, \hat{q} = 1-\hat{p}\)

daerah kritis: \$ z \textless{} -z\_\{tabel\}\$ or \$ z \textgreater{}
z\_\{tabel\}\$

    \begin{tcolorbox}[breakable, size=fbox, boxrule=1pt, pad at break*=1mm,colback=cellbackground, colframe=cellborder]
\prompt{In}{incolor}{49}{\boxspacing}
\begin{Verbatim}[commandchars=\\\{\}]
\PY{c+c1}{\PYZsh{} H0 : p1 = p2}
\PY{c+c1}{\PYZsh{} H1 : p1 != p2}
\PY{n}{x1MoreThan7} \PY{o}{=} \PY{n}{df}\PY{o}{.}\PY{n}{head}\PY{p}{(}\PY{l+m+mi}{500}\PY{p}{)}\PY{o}{.}\PY{n}{loc}\PY{p}{[}\PY{n}{df}\PY{p}{[}\PY{l+s+s1}{\PYZsq{}}\PY{l+s+s1}{alcohol}\PY{l+s+s1}{\PYZsq{}}\PY{p}{]}\PY{o}{\PYZgt{}}\PY{l+m+mi}{7}\PY{p}{,} \PY{l+s+s2}{\PYZdq{}}\PY{l+s+s2}{alcohol}\PY{l+s+s2}{\PYZdq{}}\PY{p}{]}
\PY{n}{pTopi1} \PY{o}{=} \PY{n+nb}{len}\PY{p}{(}\PY{n}{x1MoreThan7}\PY{p}{)}\PY{o}{/}\PY{l+m+mi}{1000}
\PY{n}{x2MoreThan7} \PY{o}{=} \PY{n}{df}\PY{o}{.}\PY{n}{tail}\PY{p}{(}\PY{l+m+mi}{500}\PY{p}{)}\PY{o}{.}\PY{n}{loc}\PY{p}{[}\PY{n}{df}\PY{p}{[}\PY{l+s+s1}{\PYZsq{}}\PY{l+s+s1}{alcohol}\PY{l+s+s1}{\PYZsq{}}\PY{p}{]}\PY{o}{\PYZgt{}}\PY{l+m+mi}{7}\PY{p}{,} \PY{l+s+s2}{\PYZdq{}}\PY{l+s+s2}{alcohol}\PY{l+s+s2}{\PYZdq{}}\PY{p}{]}
\PY{n}{pTopi2} \PY{o}{=} \PY{n+nb}{len}\PY{p}{(}\PY{n}{x2MoreThan7}\PY{p}{)}\PY{o}{/}\PY{l+m+mi}{1000}

\PY{n}{sig} \PY{o}{=} \PY{l+m+mf}{0.05}
\PY{c+c1}{\PYZsh{}hitung p}
\PY{n}{p} \PY{o}{=} \PY{p}{(}\PY{n+nb}{len}\PY{p}{(}\PY{n}{x1MoreThan7}\PY{p}{)}\PY{o}{+}\PY{n+nb}{len}\PY{p}{(}\PY{n}{x2MoreThan7}\PY{p}{)}\PY{p}{)}\PY{o}{/}\PY{p}{(}\PY{l+m+mi}{1000} \PY{o}{+} \PY{l+m+mi}{1000}\PY{p}{)}
\PY{c+c1}{\PYZsh{} display(p)}

\PY{c+c1}{\PYZsh{} z tabel}
\PY{n}{zTab} \PY{o}{=} \PY{n}{norm}\PY{o}{.}\PY{n}{ppf}\PY{p}{(}\PY{l+m+mi}{1}\PY{o}{\PYZhy{}}\PY{n}{sig}\PY{o}{/}\PY{l+m+mi}{2}\PY{p}{)}
\PY{n}{display}\PY{p}{(}\PY{n}{Markdown}\PY{p}{(}\PY{l+s+sa}{f}\PY{l+s+s2}{\PYZdq{}}\PY{l+s+s2}{z tabel: }\PY{l+s+si}{\PYZob{}}\PY{n+nb}{round}\PY{p}{(}\PY{n}{zTab}\PY{p}{,}\PY{l+m+mi}{3}\PY{p}{)}\PY{l+s+si}{\PYZcb{}}\PY{l+s+s2}{\PYZdq{}}\PY{p}{)}\PY{p}{)}

\PY{c+c1}{\PYZsh{}hitung z}
\PY{n}{z} \PY{o}{=} \PY{p}{(}\PY{n}{pTopi1} \PY{o}{\PYZhy{}} \PY{n}{pTopi2}\PY{p}{)}\PY{o}{/}\PY{n}{np}\PY{o}{.}\PY{n}{sqrt}\PY{p}{(}\PY{n}{p}\PY{o}{*}\PY{p}{(}\PY{l+m+mi}{1}\PY{o}{\PYZhy{}}\PY{n}{p}\PY{p}{)}\PY{o}{*}\PY{p}{(}\PY{l+m+mi}{1}\PY{o}{/}\PY{l+m+mi}{500}\PY{o}{+}\PY{l+m+mi}{1}\PY{o}{/}\PY{l+m+mi}{500}\PY{p}{)}\PY{p}{)}
\PY{n}{display}\PY{p}{(}\PY{n}{Markdown}\PY{p}{(}\PY{l+s+sa}{f}\PY{l+s+s2}{\PYZdq{}}\PY{l+s+s2}{z: }\PY{l+s+si}{\PYZob{}}\PY{n}{z}\PY{l+s+si}{\PYZcb{}}\PY{l+s+s2}{\PYZdq{}}\PY{p}{)}\PY{p}{)}

\PY{n}{PVal} \PY{o}{=} \PY{n}{norm}\PY{o}{.}\PY{n}{sf}\PY{p}{(}\PY{n}{z}\PY{p}{)}

\PY{n}{display}\PY{p}{(}\PY{n}{Markdown}\PY{p}{(}\PY{l+s+sa}{f}\PY{l+s+s2}{\PYZdq{}}\PY{l+s+s2}{Nilai p = }\PY{l+s+si}{\PYZob{}}\PY{n+nb}{round}\PY{p}{(}\PY{n}{PVal}\PY{p}{,}\PY{l+m+mi}{3}\PY{p}{)}\PY{l+s+si}{\PYZcb{}}\PY{l+s+s2}{\PYZdq{}}\PY{p}{)}\PY{p}{)}

\PY{k}{if} \PY{p}{(}\PY{n}{PVal}\PY{o}{\PYZlt{}}\PY{l+m+mf}{0.05} \PY{o+ow}{and} \PY{p}{(}\PY{n}{z} \PY{o}{\PYZlt{}} \PY{o}{\PYZhy{}}\PY{n}{zTab} \PY{o+ow}{or} \PY{n}{z} \PY{o}{\PYZgt{}} \PY{n}{zTab}\PY{p}{)}\PY{p}{)}\PY{p}{:}
    \PY{n+nb}{print}\PY{p}{(}\PY{l+s+s2}{\PYZdq{}}\PY{l+s+s2}{H0 ditolak : p1 = p2}\PY{l+s+s2}{\PYZdq{}}\PY{p}{)}
\PY{k}{else}\PY{p}{:}
    \PY{n+nb}{print}\PY{p}{(}\PY{l+s+s2}{\PYZdq{}}\PY{l+s+s2}{H1 diterima : p1 != p2}\PY{l+s+s2}{\PYZdq{}}\PY{p}{)}
\end{Verbatim}
\end{tcolorbox}

    z tabel: 1.96

    
    z: 0.0

    
    Nilai p = 0.5

    
    \begin{Verbatim}[commandchars=\\\{\}]
H1 diterima : p1 != p2
    \end{Verbatim}


    % Add a bibliography block to the postdoc
    
    
    
\end{document}
